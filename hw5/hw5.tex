\documentclass[12pt]{article}


%----------Packages----------
\usepackage{amsmath}
\usepackage{amssymb}
\usepackage{amsthm}
%\usepackage{amsrefs}
\usepackage{dsfont}
\usepackage{mathrsfs}
\usepackage{stmaryrd}
\usepackage[all]{xy}
\usepackage[mathcal]{eucal}
\usepackage{verbatim}  %%includes comment environment
\usepackage{fullpage}  %%smaller margins
\usepackage{hyperref}
\usepackage{setspace}
\onehalfspacing
%----------Commands----------

%%penalizes orphans 
\clubpenalty=9999
\widowpenalty=9999 

%% bold math capitals
\newcommand{\bA}{\mathbf{A}}
\newcommand{\bB}{\mathbf{B}}
\newcommand{\bC}{\mathbf{C}}
\newcommand{\bD}{\mathbf{D}}
\newcommand{\bE}{\mathbf{E}}
\newcommand{\bF}{\mathbf{F}}
\newcommand{\bG}{\mathbf{G}}
\newcommand{\bH}{\mathbf{H}}
\newcommand{\bI}{\mathbf{I}}
\newcommand{\bJ}{\mathbf{J}}
\newcommand{\bK}{\mathbf{K}}
\newcommand{\bL}{\mathbf{L}}
\newcommand{\bM}{\mathbf{M}}
\newcommand{\bN}{\mathbf{N}}
\newcommand{\bO}{\mathbf{O}}
\newcommand{\bP}{\mathbf{P}}
\newcommand{\bQ}{\mathbf{Q}}
\newcommand{\bR}{\mathbf{R}}
\newcommand{\bS}{\mathbf{S}}
\newcommand{\bT}{\mathbf{T}}
\newcommand{\bU}{\mathbf{U}}
\newcommand{\bV}{\mathbf{V}}
\newcommand{\bW}{\mathbf{W}}
\newcommand{\bX}{\mathbf{X}}
\newcommand{\bY}{\mathbf{Y}}
\newcommand{\bZ}{\mathbf{Z}}

%% blackboard bold math capitals
\newcommand{\bbA}{\mathbb{A}}
\newcommand{\bbB}{\mathbb{B}}
\newcommand{\bbC}{\mathbb{C}}
\newcommand{\bbD}{\mathbb{D}}
\newcommand{\bbE}{\mathbb{E}}
\newcommand{\bbF}{\mathbb{F}}
\newcommand{\bbG}{\mathbb{G}}
\newcommand{\bbH}{\mathbb{H}}
\newcommand{\bbI}{\mathbb{I}}
\newcommand{\bbJ}{\mathbb{J}}
\newcommand{\bbK}{\mathbb{K}}
\newcommand{\bbL}{\mathbb{L}}
\newcommand{\bbM}{\mathbb{M}}
\newcommand{\bbN}{\mathbb{N}}
\newcommand{\bbO}{\mathbb{O}}
\newcommand{\bbP}{\mathbb{P}}
\newcommand{\bbQ}{\mathbb{Q}}
\newcommand{\bbR}{\mathbb{R}}
\newcommand{\bbS}{\mathbb{S}}
\newcommand{\bbT}{\mathbb{T}}
\newcommand{\bbU}{\mathbb{U}}
\newcommand{\bbV}{\mathbb{V}}
\newcommand{\bbW}{\mathbb{W}}
\newcommand{\bbX}{\mathbb{X}}
\newcommand{\bbY}{\mathbb{Y}}
\newcommand{\bbZ}{\mathbb{Z}}

%% script math capitals
\newcommand{\sA}{\mathscr{A}}
\newcommand{\sB}{\mathscr{B}}
\newcommand{\sC}{\mathscr{C}}
\newcommand{\sD}{\mathscr{D}}
\newcommand{\sE}{\mathscr{E}}
\newcommand{\sF}{\mathscr{F}}
\newcommand{\sG}{\mathscr{G}}
\newcommand{\sH}{\mathscr{H}}
\newcommand{\sI}{\mathscr{I}}
\newcommand{\sJ}{\mathscr{J}}
\newcommand{\sK}{\mathscr{K}}
\newcommand{\sL}{\mathscr{L}}
\newcommand{\sM}{\mathscr{M}}
\newcommand{\sN}{\mathscr{N}}
\newcommand{\sO}{\mathscr{O}}
\newcommand{\sP}{\mathscr{P}}
\newcommand{\sQ}{\mathscr{Q}}
\newcommand{\sR}{\mathscr{R}}
\newcommand{\sS}{\mathscr{S}}
\newcommand{\sT}{\mathscr{T}}
\newcommand{\sU}{\mathscr{U}}
\newcommand{\sV}{\mathscr{V}}
\newcommand{\sW}{\mathscr{W}}
\newcommand{\sX}{\mathscr{X}}
\newcommand{\sY}{\mathscr{Y}}
\newcommand{\sZ}{\mathscr{Z}}

\renewcommand{\phi}{\varphi}
%\renewcommand{\emptyset}{\O}

\providecommand{\abs}[1]{\lvert #1 \rvert}
\providecommand{\norm}[1]{\lVert #1 \rVert}
\providecommand{\x}{\times}
\providecommand{\ar}{\rightarrow}
\providecommand{\arr}{\longrightarrow}


%----------Theorems----------

\newtheorem{theorem}{Theorem}[section]
\newtheorem{proposition}[theorem]{Proposition}
\newtheorem{lemma}[theorem]{Lemma}
\newtheorem{corollary}[theorem]{Corollary}

\theoremstyle{definition}
\newtheorem{definition}[theorem]{Definition}
\newtheorem{nondefinition}[theorem]{Non-Definition}
\newtheorem{exercise}[theorem]{Exercise}

\numberwithin{equation}{subsection}


%----------Title-------------
\title{Homework 5}
\author{Yifan}

\begin{document}

\pagestyle{plain}


%%---  sheet number for theorem counter
%\setcounter{section}{1}

\begin{center}
{\large Homework 5} \\
\vspace{.2in}
Yifan Liu
\end{center}

\bigskip \bigskip

% \section{Chapter 1}

\textbf{Exercise 1} $\text { Let } S=\left\{1-\frac{(-1)^{n}}{n} 
\mid n \in \mathbb{N}\right\} . 
\text { Find the infimum and supremum of } S$.
\begin{center}
    inf $S = \frac{1}{2}$, sup $S = 2$
\end{center}
\begin{proof}
If $n$ is odd, $1-\frac{(-1)^{n}}{n} = 1+\frac{1}{n}$;
If $n$ is even, $1-\frac{(-1)^{n}}{n} = 1-\frac{1}{n}$.

First, $\frac{1}{2}$ is a lower bound; $2$ is an upper bound.\\
If $n$ is odd, $1-\frac{(-1)^{n}}{n} = 1+\frac{1}{n} > 1 > \frac{1}{2}$; If $n$ is even ($n \geq 2$), $1-\frac{(-1)^{n}}{n} = 1-\frac{1}{n} \geq 1-\frac{1}{2} = \frac{1}{2}$. Since a natural number is either odd or even, the elements in $S$ is greater than $\frac{1}{2}$ anyway. $\frac{1}{2}$ is a lower bound of $S$.\\
If $n$ is even, $1-\frac{(-1)^{n}}{n} = 1-\frac{1}{n} < 1 < 2$; If $n$ is odd ($n \geq 1$), $1-\frac{(-1)^{n}}{n} = 1+\frac{1}{n} < 1+\frac{1}{1} = 2$.
Since a natural number is either odd or even, the elements in $S$ is less than $2$ anyway. $2$ is an upper bound of $S$.

Second, any other lower bound is less than or equal to $\frac{1}{2}$; any other upper bound is greater than or equal to $2$.\\
This is because $\frac{1}{2}$ and $2$ are elements in $S$ ($\frac{1}{2}$ when $n =2$; $2$ when $n=1$). By definition, any lower bound should be less than or equal to all elements in $S$, including $\frac{1}{2}$; any upper bound should be greater than or equal to all elements in $S$, including $2$.
    
Hence, inf $S = \frac{1}{2}$, sup $S = 2$.
\end{proof}

\textbf{Exercise 2} Show that sup$\left\{1-\frac{1}{n} 
\mid n \in \mathbb{N}\right\}=1$.
\begin{proof}
Let $A = \left\{1-\frac{1}{n} 
\mid n \in \mathbb{N}\right\}=1$.\\
First, $1$ is an upper bound of $A$: because $\frac{1}{n} > 0 \text{ for all }
n \in \bbN$, $1-\frac{1}{n} < 1-0=1$. \\
Second, if $M$ is an upper bound of $A$, $M \geq 1$. In other words, if $M < 1$, $M$ is not an upper bound. Let's prove this: Let $\epsilon = 1 - M > 0$. There exists $N \in \bbN$ s.t. $N > \frac{1}{\epsilon}$. Then, $1-\frac{1}{N} > 1-\epsilon = M$. So $M$ is not an upper bound of $A$.\\
Consequently, $1$ is the supremum of $A$.
\end{proof}

\textbf{Exercise 3} Let $S$ be a nonempty subset of $\bbR$ that is bounded below. Prove that
\\inf $S=-\text{sup}\left\{-s \mid s \in S \right\}$.
\newcommand{\infS}{\text{inf } S}
\newcommand{\supS}{\text{sup } S'}
\begin{proof}
Let $S' = \left\{-s \mid s \in S \right\}$. We want to prove that $-\infS$ is the supremum of $S'$.

1) By definition, $\forall s \in S, s \geq \infS$. So $\forall s \in S, -s \leq -\infS$, which is equivalent to $\forall s' \in S', s' \leq -\infS$. Therefore, $-\infS$ is an upper bound of $S'$.

2) If $t$ is an upper bound of $S'$, then $t \geq -\infS$:
$$
\forall s' \in S', s' \leq t
$$
So
$$
\forall s' \in S', -s' \geq -t
$$
According to the definition of $S'$,
$$
\forall s \in S, s \geq -t
$$
This is to say, $-t$ is a lower bound of $S$, so it should be less than or equal to the least lower bound of $S$. $-t\leq \infS$, so $t \geq -\infS$.

Therefore, $-\infS$ is the supremum of $S'$, which is the same to say, inf $S=-\text{sup}\left\{-s \mid s \in S \right\}$.
\end{proof}

\textbf{Exercise 4} Suppose that a set $S \subset \bbR$ contains one of
 its upper bounds. Show that this upper bound is the supremum.
\begin{proof}
Let this upper bound in $S$ be $L$.\\
First, $L$ is an upper bound of $S$ (given);\\
Second, if $M$ is an upper bound of $S$, then it should be greater than any
element in $S$, including $L$. In other words, $L < M$.\\
By definition, this shows that this upper bound $L$ is the supremum. 
\end{proof}

\textbf{Exercise 5} Prove the Archimedean property: that if 
$x \in \bbR$ then $\exists n_x \in \bbN$ so that
$x < n_x$.
\begin{proof}
Suppose for a given $x \in \bbR$ there does NOT exist such a $n_x > x$, 
in other words, $\forall n \in \bbN, x > n$. Then $x$ is an upper bound of 
$\bbN$ in $\bbR$. By the completeness of $\bbR$ (or that least upper bounds exist in $\bbR$), there exists a least upper bound $L$. Hence, $L - 1$ should not be an upper bound of $\bbN$, which implies that there exists $n \in \bbN$ such that $L-1<n$, but this means that $L < n+1$ where $n+1 \in \bbN$, which contradicts with $L$ being the upper bound of $\bbN$. Therefore, the assumption must be false; there exists $n_x \in \bbN$ so that $x < n_x$.
\end{proof}

\textbf{Exercise 3.1.12} Let $n$ be a positive integer that is not a perfect square.
Let $A=\left\{x \in \mathbb{Q} \mid x^{2}<n\right\} .$ Show that $A$ is bounded in $\bbQ$ but has neither
a glb nor a lub in $\bbQ$. Conclude that $\sqrt{n}$ exists in $\mathbb{R},$ that is, there exists a
real number $a$ such that $a^{2}=n .$
\begin{proof}

\end{proof}

% Exercise $3.1 .14 .$ Suppose that $A$ and $B$ are bounded sets in $\mathbb{R} .$ Prove or
% disprove the following:
% $$
% \begin{aligned}(i) \ln \mathrm{b}(A \cup B)=\max \{\operatorname{lub}(A), \operatorname{lub}(B)\} \\ \text { (ii) } \operatorname{lf} A+B=\{a+b \mid a \in A, b \in B\}, \text { then } \operatorname{lub}(A+B)=\operatorname{lub}(A)+\\ \operatorname{lub}(B) \end{aligned}
% $$
% (iii) If the elements of $A$ and $B$ are positive and $A \cdot B=\{a b \mid a \in A, b \in$
% $\mathrm{~ B \} , ~ t h e n ~} \operatorname{lub}(A \cdot B)=\operatorname{lub}(A) \operatorname{lub}(B)$
% (iv) Formulate the analogous problems for the greatest lower bound.

\textbf{Exercise 3.1.14} Suppose that $A$ and $B$ are bounded sets in $\mathbb{R} .$ Prove or
disprove the following:

(i) $\text{lub} (A \cup B)=\max \{\operatorname{lub}(A), \operatorname{lub}(B)\} $

(ii) If $A+B=\{a+b \mid a \in A, b \in B\}, \text { then } \operatorname{lub}(A+B)=\operatorname{lub}(A)+ \operatorname{lub}(B) $

(iii) If the elements of $A$ and $B$ are positive and $A \cdot B=\{a b \mid a \in A, b \in$
$\mathrm{~ B \} , ~ t h e n ~} \operatorname{lub}(A \cdot B)=\operatorname{lub}(A) \operatorname{lub}(B)$

(iv) Formulate the analogous problems for the greatest lower bound.
\begin{proof}

\end{proof}


\textbf{Exercise 3.6.3} Prove the properties of the absolute value.

(1) For any $x \in \mathbb{R},|x| \geq 0,$ and $|x|=0$ iff $x=0$.

(2) For any $x, y \in \mathbb{R},|x y|=|x||y|$.

(3) For any $x, y \in \mathbb{R},|x+y| \leq|x|+|y|$ (triangle inequality).


\begin{proof}
(1) For $x \in \bbR$, either $x \geq 0$ or $x < 0$. Let's prove that $|x| \geq 0$ in both cases. If $x \geq 0$, $|x| = x \geq 0$; if $x < 0$, $|x| = -x > 0$. So for any  $x \in \mathbb{R},|x| \geq 0$. 

If $x=0$, because $x\geq 0$, $|x|=x =0$. We then prove the contrapositive of $|x|=0\implies x=0$. If $x\neq 0$, it's either greater or less than 0. If greater than 0, then $|x|=x>0\neq 0$; if less than 0, then $|x|=-x>0\neq 0$. Combining two directions, we have $|x|=0$ iff $x=0$.

\bigskip
(2) 
If $x=0$ or $y=0$
then:
$
xy=0 \implies |xy|=|0|=0
$
and either $|x|=0$ or $|y|=0$ so 
$
|x||y|=0=|xy|
$.

If $x>0$ and $y>0$
then:
$|x| =x, |y| =y \implies |x||y| =x y $;
and 
$xy>0, |xy|=xy = |x||y|$.

If $x<0$ and $y<0$
then:
$|x| =-x, |y| =-y \implies |x||y| =(-x)(-y)=xy $;
and 
$xy>0, |xy|=xy = |x||y|$.

If $x<0$ and $y>0$
then:
$|x| =-x, |y| =y \implies |x||y| =(-x)y=-xy $;
and 
$xy<0, |xy|=-xy = |x||y|$.

If $x>0$ and $y<0$
then:
$|x| =x, |y| =-y \implies |x||y| =x(-y)=-xy $;
and 
$xy<0, |xy|=-xy = |x||y|$.

Hence, in all possible cases, $|xy|=|x||y|$.

\bigskip
(3) We first prove that $x\leq|x|$ and $-x\leq|x|$ for all $x\in \bbR$.

If $x\geq 0$, then $|x|=x\geq 0$, $x\leq x=|x|$ and $-x\leq 0 \leq|x|$;

If $x < 0$, then $|x|=-x > 0$, $x < 0 \leq|x|$ and $-x = |x| \leq|x|$.

With this, we prove (3):
$$
\begin{aligned} a+b \leq|a|+b \leq|a|+|b| \\-a-b \leq|a|-b \leq|a|+|b| \end{aligned}
$$

Since $a+b$, depending on its sign, is either $a+b$ or $-a-b$, while both are less than or equal to $|a|+|b|$, we can conclude that $|x+y| \leq|x|+|y|$.
\end{proof}


\textbf{Exercise 3.6.5} Show that the limit of a convergent sequence is unique.
\begin{proof}
Let the convergent sequence be $(a_k)_{k\in \bbN}$. 
Suppose it has two different limits $a$ and $b$ $(a \neq b)$.
Let $\epsilon = \frac{|a-b|}{5}$.\\
According to the definition of convergent sequence, there exists $N_1 \in \bbN$ such that 
$|a_n - a| < \epsilon$ for all $n > N_1$; 
Also
there exists $N_2 \in \bbN$ such that 
$|a_n - b| < \epsilon$ for all $n > N_1$.\\
Let $N$ be the greater one of $N_1$ and $N_2$, then 
$|a_n - a| < \epsilon$ and 
$|a_n - b| < \epsilon$ for all $n > N$.\\
However, by triangle inequality, $|a-b| \leq |a-a_n|+|a_n-b| = |a_n-a|+|a_n-b| \leq \frac{2|a-b|}{5}$, which is impossible ($1$ is greater than $\frac{2}{5}$). Contradiction! \\Hence the assumption $a\neq b$ is false. $a=b$. In other words, the limit of the convergent sequence is unique. 
\end{proof}


\end{document}