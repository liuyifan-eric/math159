\documentclass[12pt]{article}


%----------Packages----------
\usepackage{amsmath}
\usepackage{amssymb}
\usepackage{amsthm}
%\usepackage{amsrefs}
\usepackage{dsfont}
\usepackage{mathrsfs}
\usepackage{stmaryrd}
\usepackage[all]{xy}
\usepackage[mathcal]{eucal}
\usepackage{verbatim}  %%includes comment environment
\usepackage{fullpage}  %%smaller margins
\usepackage{hyperref}
\usepackage{setspace}
\onehalfspacing
%----------Commands----------

%%penalizes orphans
\clubpenalty=9999
\widowpenalty=9999

%% bold math capitals
\newcommand{\bA}{\mathbf{A}}
\newcommand{\bB}{\mathbf{B}}
\newcommand{\bC}{\mathbf{C}}
\newcommand{\bD}{\mathbf{D}}
\newcommand{\bE}{\mathbf{E}}
\newcommand{\bF}{\mathbf{F}}
\newcommand{\bG}{\mathbf{G}}
\newcommand{\bH}{\mathbf{H}}
\newcommand{\bI}{\mathbf{I}}
\newcommand{\bJ}{\mathbf{J}}
\newcommand{\bK}{\mathbf{K}}
\newcommand{\bL}{\mathbf{L}}
\newcommand{\bM}{\mathbf{M}}
\newcommand{\bN}{\mathbf{N}}
\newcommand{\bO}{\mathbf{O}}
\newcommand{\bP}{\mathbf{P}}
\newcommand{\bQ}{\mathbf{Q}}
\newcommand{\bR}{\mathbf{R}}
\newcommand{\bS}{\mathbf{S}}
\newcommand{\bT}{\mathbf{T}}
\newcommand{\bU}{\mathbf{U}}
\newcommand{\bV}{\mathbf{V}}
\newcommand{\bW}{\mathbf{W}}
\newcommand{\bX}{\mathbf{X}}
\newcommand{\bY}{\mathbf{Y}}
\newcommand{\bZ}{\mathbf{Z}}

%% blackboard bold math capitals
\newcommand{\bbA}{\mathbb{A}}
\newcommand{\bbB}{\mathbb{B}}
\newcommand{\bbC}{\mathbb{C}}
\newcommand{\bbD}{\mathbb{D}}
\newcommand{\bbE}{\mathbb{E}}
\newcommand{\bbF}{\mathbb{F}}
\newcommand{\bbG}{\mathbb{G}}
\newcommand{\bbH}{\mathbb{H}}
\newcommand{\bbI}{\mathbb{I}}
\newcommand{\bbJ}{\mathbb{J}}
\newcommand{\bbK}{\mathbb{K}}
\newcommand{\bbL}{\mathbb{L}}
\newcommand{\bbM}{\mathbb{M}}
\newcommand{\bbN}{\mathbb{N}}
\newcommand{\bbO}{\mathbb{O}}
\newcommand{\bbP}{\mathbb{P}}
\newcommand{\bbQ}{\mathbb{Q}}
\newcommand{\bbR}{\mathbb{R}}
\newcommand{\bbS}{\mathbb{S}}
\newcommand{\bbT}{\mathbb{T}}
\newcommand{\bbU}{\mathbb{U}}
\newcommand{\bbV}{\mathbb{V}}
\newcommand{\bbW}{\mathbb{W}}
\newcommand{\bbX}{\mathbb{X}}
\newcommand{\bbY}{\mathbb{Y}}
\newcommand{\bbZ}{\mathbb{Z}}

%% script math capitals
\newcommand{\sA}{\mathscr{A}}
\newcommand{\sB}{\mathscr{B}}
\newcommand{\sC}{\mathscr{C}}
\newcommand{\sD}{\mathscr{D}}
\newcommand{\sE}{\mathscr{E}}
\newcommand{\sF}{\mathscr{F}}
\newcommand{\sG}{\mathscr{G}}
\newcommand{\sH}{\mathscr{H}}
\newcommand{\sI}{\mathscr{I}}
\newcommand{\sJ}{\mathscr{J}}
\newcommand{\sK}{\mathscr{K}}
\newcommand{\sL}{\mathscr{L}}
\newcommand{\sM}{\mathscr{M}}
\newcommand{\sN}{\mathscr{N}}
\newcommand{\sO}{\mathscr{O}}
\newcommand{\sP}{\mathscr{P}}
\newcommand{\sQ}{\mathscr{Q}}
\newcommand{\sR}{\mathscr{R}}
\newcommand{\sS}{\mathscr{S}}
\newcommand{\sT}{\mathscr{T}}
\newcommand{\sU}{\mathscr{U}}
\newcommand{\sV}{\mathscr{V}}
\newcommand{\sW}{\mathscr{W}}
\newcommand{\sX}{\mathscr{X}}
\newcommand{\sY}{\mathscr{Y}}
\newcommand{\sZ}{\mathscr{Z}}

\renewcommand{\phi}{\varphi}
%\renewcommand{\emptyset}{\O}

\providecommand{\abs}[1]{\lvert #1 \rvert}
\providecommand{\norm}[1]{\lVert #1 \rVert}
\providecommand{\x}{\times}
\providecommand{\ar}{\rightarrow}
\providecommand{\arr}{\longrightarrow}


%----------Theorems----------

\newtheorem{theorem}{Theorem}[section]
\newtheorem{proposition}[theorem]{Proposition}
\newtheorem{lemma}[theorem]{Lemma}
\newtheorem{corollary}[theorem]{Corollary}

\theoremstyle{definition}
\newtheorem{definition}[theorem]{Definition}
\newtheorem{nondefinition}[theorem]{Non-Definition}
\newtheorem{exercise}[theorem]{Exercise}

\numberwithin{equation}{subsection}


%----------Title-------------
\title{Homework 6}
\author{Yifan}

\begin{document}

\pagestyle{plain}


%%---  sheet number for theorem counter
%\setcounter{section}{1}

\begin{center}
{\large Homework 6} \\
\vspace{.2in}
Yifan Liu
\end{center}

\bigskip \bigskip

% \section{Chapter 1}

\textbf{Exercise 3.5.1} Show that, for any $a, b \in \mathbb{Q},$ we have $|| a|-| b|| \leq|a-b|$ 
\begin{proof}
$$
|a| = |a - b + b| \leq |a - b| + |b| \implies |a| - |b| \leq |a - b|
$$
$$|b| =|b-a+a| \leq|a-b|+|a| \\ \Longrightarrow|b|-|a|  \leq|a-b| \\ \Longrightarrow-|a-b|  \leq|a|-|b| $$
Hence, $|| a|-| b|| \leq|a-b|$.
\end{proof}


\textbf{Exercise 2} Give an example showing that Theorem 3.4.4 in the textbook
is false if we remove the assumption that:

(a) the intervals are closed.

(b) the intervals are nested.

(Theorem 3.4.4 (Nested Intervals Theorem). Let $\left(\left[a_{n}, b_{n}\right]\right)_{n \in \mathbb{N}}$ be a nested
sequence of closed bounded intervals in $\mathbb{R}$. That is, for any $n,$ we have
$\left[a_{n+1}, b_{n+1}\right] \subseteq\left[a_{n}, b_{n}\right],$ or equivalently, $a_{n} \leq a_{n+1} \leq b_{n+1} \leq b_{n}$ for all $n$.
Then $\bigcap_{n \in \mathbb{N}}\left[a_{n}, b_{n}\right] \neq \varnothing .$)
\begin{proof} Example:

    (a) $a_n = 0, b_n = 0$. Apparently $\bigcap_{n \in \mathbb{N}}\left[a_{n}, b_{n}\right] =\varnothing$.

    Or a normal version: $a_n = \frac{1}{n}, b_n = 0$. Any element in the intersect should be greater than $b_n = 0$, yet we can always find a $a_n = \frac{1}{n}$ by Archimedean property that is smaller than such element, so such elem does not exist, the intersect is emptyset. 

    (b) $a_n = \frac{1}{n}, b_n = \frac{1}{n} + \frac{1}{2}$, then $\bigcap_{n \in \mathbb{N}}\left[a_{n}, b_{n}\right] = \varnothing $. Because if there is an element, it should be greater than $a_1 = 1$ but less than or equal to the limit of $b_n$, $\frac{1}{2}$, which is impossible.  
\end{proof}


\textbf{Exercise 3} Let $b \in \mathbb{R} .$ Prove that $(b / n)_{n \in \mathbb{N}} \rightarrow 0$
\begin{proof}
For a given $\epsilon$, by Archimedean property, there exists a $N \in \bbN$ s.t. $N\cdot\epsilon > |b|$. Then, $\forall n > N, \left|\frac{b}{n} - 0\right|  = \frac{|b|}{n} < \frac{|b|}{N} < \epsilon$. So the sequence converges to $0$. 
\end{proof}


\textbf{Exercise 4} Let $\left(b_{k}\right)_{k \in \mathbb{N}} \rightarrow m$ and suppose that $b_{k}>0$ for every $k \in \mathbb{N}$ and that $m>0 .$

(a) Show that there exists an $N_{1} \in \mathbb{N}$ such that
$$
b_{k} \geq m / 2 \text { for all } k \geq N_{1}
$$

(b) Show that $\left(1 / b_{k}\right)_{k \in \mathbb{N}}$ converges to $1 / m$.
\begin{proof}
    (a) Let $\epsilon$ be $m/2$, then there exists a $N_1 \in \bbN$ s.t. for all $k \geq N_1$, $|b_k - m| < \epsilon = m/2$, since the sequence converges to $m$. Therefore, $b_k > m - m/2 = m/2$. 
    
    (b) Since $m > 0$, there must be a lower bound of the sequence that is greater than $0$. (Take $\epsilon = m/2$, so there is a $N$ s.t. $\forall n > N, |b_n - m| < m/2$. Find the minimum of $b_i$ for $i < N$. Then the smaller one of this minimum and $m/2$ is a positive lower bound.) Let $M$ be this bound, then $|b_k| 
    \geq M$ for all $k$. 
        
    Given $\epsilon$, consider $mM\epsilon $. $\exists N' \in \bbN $ s.t. $\forall k > N', |b_k - m| < mM\epsilon$. Therefore, $$
    \left|\frac{1}{b_k}-\frac{1}{m}\right| = \left|\frac{m-b_k}{b_km}\right| < \frac{mM\epsilon}{b_km} \leq \frac{mM\epsilon}{mM} =\epsilon
    $$

    Hence, $\left(1 / b_{k}\right)_{k \in \mathbb{N}}$ converges to $1 / m$.

\end{proof}


\textbf{Exercise 5} Show that
$$
n^3-6 \geq n^2
$$

for  all sufficiently large $n \in \bbN$. 
\begin{proof}
$$
n^3-6 \geq n^2 \Longleftrightarrow n^2(n-1) \geq 6
$$
Because $n\geq 1 \implies n^2 \geq  1$, when $n\geq 7$, 
$$
n^2(n-1) \geq 1\cdot(7-1) =6
$$
So the inequality holds at least for all $n \geq 7$ (sufficiently large).
\end{proof}


\textbf{Exercise 6} Prove  that the sequence

$$
\left(\frac{4 n^{3}+3 n}{n^{3}-6}\right)_{n \in \mathbb{N}}
$$

converges in $\bbR$.
\begin{proof}
    Given $\epsilon > 0$, let $N$ be the first natural number greater than or equal to $ \max\{48/\epsilon, 7\}$ ($N$ exists because of Archimedean property). Then for any $n > N$ , ($n^3-6\geq n^2 > n > 0 $ from previous question):
    $$
    \left|\frac{4 n^{3}+3 n}{n^{3}-6}-4\right| = \left|\frac{3 n +  24}{n^{3}-6}\right| = \frac{3 n +  24}{n^{3}-6} \leq \frac{3 n +  24}{n^2} = \frac{3 n }{n^2} +\frac{24}{n^2} < \frac{3}{N} + \frac{24}{N} \leq \frac{3}{48/\epsilon} + \frac{24}{48/\epsilon} < \epsilon. 
    $$
    Therefore the sequence converges in $\bbR$ to $4$. 
\end{proof}



\textbf{Exercise 7} Prove that the sequence

$$
\left(\frac{12 n^{5}+3 n^{4}}{2 n^{5}-1}\right)_{n \in \mathbb{N}}
$$

converges in $\bbR$.
\begin{proof}
    Given $\epsilon > 0$, let $N$ be the first natural number greater than or equal to $ 12/\epsilon$ ($N$ exists because of Archimedean property). Then, for $n > N$, because $2n^5>1, n^5 \geq n \geq 1$ for $n \in \bbN$, we have:
    $$
    \left|\frac{12n^5+3n^4}{2n^5-1}-6\right| = \left|\frac{3n^4+6}{2n^5-1}\right| = \frac{3n^4+6}{2n^5-1} \leq \frac{3n^4+6}{2n^5-n^5} = \frac{3}{n} + \frac{6}{n^5} < \frac{3}{N} + \frac{6}{N} \leq \frac{3}{12/\epsilon} + \frac{6}{12/\epsilon} < \epsilon
    $$
    So the sequence converges to $6$ in $\bbR$. 
\end{proof}

\textbf{Exercise 8} Let $\left(s_{n}\right)_{n \in \mathbb{N}}$ be a sequence of strictly positive real numbers and
suppose that it converges to $s \in \mathbb{R}$.

(a) If $s=0,$ show that $(\sqrt{s_{n}})_{n \in \mathbb{N}} \rightarrow 0$


(b) If $s>0,$ show that $(\sqrt{s_{n}})_{n \in \mathbb{N}} \rightarrow \sqrt{s}$
\begin{proof}
    Apparently $(\sqrt{s_{n}})_{n \in \mathbb{N}}$ is decreasing and is greater than 0, so the limit exists. 

Using the definition of the convergence of $s_n$, and that the sequence is of strictly positive real numbers:

(a) $s=0 .$ For a given $\epsilon>0,$ we can find a $N$ s.t. if $n>N,$ then $\left|a_{n}-0\right|<\epsilon^{2} .$ Thus, $|\sqrt{s_{n}}-0|<\epsilon .$ This shows that $\sqrt{s_{n}} \rightarrow 0$.


(b) $s>0 .$ For a given $\epsilon>0,$ we can find a $N$ s.t. if $n>N,$ then $\left|s_{n}-s\right|<\epsilon \sqrt{s} .$ Thus,
$$|\sqrt{s_{n}}-\sqrt{s}|=\frac{\left|s_{n}-s\right|}{\sqrt{s_{n}}+\sqrt{s}}<\frac{\left|s_{n}-s\right|}{\sqrt{s}}<\frac{\epsilon \sqrt{s}}{\sqrt{s}}=\epsilon .$$
This shows that $\sqrt{s_{n}} \rightarrow \sqrt{s}$.

\end{proof}


\textbf{Exercise 9} $\text { Prove that }\left(x_{n}\right)_{n \in \mathbb{N}} \text { converges to } 0 $ if and only if $\left(\left|x_{n}\right|\right)_{n \in \mathbb{N}}$converges to zero. Give an example showing that convergence of $\left(\left|x_{n}\right|\right)_{n \in \mathbb{N}}$ in general does not imply convergence of $\left(x_{n}\right)_{n \in \mathbb{N}}$
\begin{proof}

    Left to right:

    For a given $\epsilon>0,$ because $x_n$ converges, $\exists N \in \mathbb{N}\left(\forall n \in \mathbb{N}\left(n \geq N \Longrightarrow\left|x_{n}-0\right|<\epsilon\right)\right)$.
    $\left|x_{n}-0\right| < \epsilon \implies |x_n| < \epsilon \implies \left|\left|x_{n}\right|-0\right| < \epsilon$. So with this same $N$ we can show the convergence of $|x_n|$. 

    Right to left:

    For a given $\epsilon>0,$ because $|x_n|$ converges, $\exists N \in \mathbb{N}\left(\forall n \in \mathbb{N}\left(n \geq N \Longrightarrow\left||x_{n}|-0\right|<\epsilon\right)\right)$.
    $\left||x_{n}|-0\right| < \epsilon \implies |x_n| < \epsilon \implies \left|x_{n}-0\right| < \epsilon$. So with this same $N$ we can show the convergence of $x_n$. 

\end{proof}

Example: $$
x_n = (-1)^n(\frac{1}{n}+1)
$$

$|x_n|$ apparently converges to 1, but $x_n$ flips between $1$ and $-1$. 

\end{document}

$\forall \epsilon>0\left(\exists N \in \mathbb{N}\left(\forall n \in \mathbb{N}\left(n \geq N \Longrightarrow\left|x_{n}-x\right|<\epsilon\right)\right)\right)$