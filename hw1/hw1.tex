\documentclass[12pt]{article}


%----------Packages----------
\usepackage{amsmath}
\usepackage{amssymb}
\usepackage{amsthm}
%\usepackage{amsrefs}
\usepackage{dsfont}
\usepackage{mathrsfs}
\usepackage{stmaryrd}
\usepackage[all]{xy}
\usepackage[mathcal]{eucal}
\usepackage{verbatim}  %%includes comment environment
\usepackage{fullpage}  %%smaller margins
\usepackage{hyperref}
\usepackage{setspace}
\onehalfspacing
%----------Commands----------

%%penalizes orphans
\clubpenalty=9999
\widowpenalty=9999

%% bold math capitals
\newcommand{\bA}{\mathbf{A}}
\newcommand{\bB}{\mathbf{B}}
\newcommand{\bC}{\mathbf{C}}
\newcommand{\bD}{\mathbf{D}}
\newcommand{\bE}{\mathbf{E}}
\newcommand{\bF}{\mathbf{F}}
\newcommand{\bG}{\mathbf{G}}
\newcommand{\bH}{\mathbf{H}}
\newcommand{\bI}{\mathbf{I}}
\newcommand{\bJ}{\mathbf{J}}
\newcommand{\bK}{\mathbf{K}}
\newcommand{\bL}{\mathbf{L}}
\newcommand{\bM}{\mathbf{M}}
\newcommand{\bN}{\mathbf{N}}
\newcommand{\bO}{\mathbf{O}}
\newcommand{\bP}{\mathbf{P}}
\newcommand{\bQ}{\mathbf{Q}}
\newcommand{\bR}{\mathbf{R}}
\newcommand{\bS}{\mathbf{S}}
\newcommand{\bT}{\mathbf{T}}
\newcommand{\bU}{\mathbf{U}}
\newcommand{\bV}{\mathbf{V}}
\newcommand{\bW}{\mathbf{W}}
\newcommand{\bX}{\mathbf{X}}
\newcommand{\bY}{\mathbf{Y}}
\newcommand{\bZ}{\mathbf{Z}}

%% blackboard bold math capitals
\newcommand{\bbA}{\mathbb{A}}
\newcommand{\bbB}{\mathbb{B}}
\newcommand{\bbC}{\mathbb{C}}
\newcommand{\bbD}{\mathbb{D}}
\newcommand{\bbE}{\mathbb{E}}
\newcommand{\bbF}{\mathbb{F}}
\newcommand{\bbG}{\mathbb{G}}
\newcommand{\bbH}{\mathbb{H}}
\newcommand{\bbI}{\mathbb{I}}
\newcommand{\bbJ}{\mathbb{J}}
\newcommand{\bbK}{\mathbb{K}}
\newcommand{\bbL}{\mathbb{L}}
\newcommand{\bbM}{\mathbb{M}}
\newcommand{\bbN}{\mathbb{N}}
\newcommand{\bbO}{\mathbb{O}}
\newcommand{\bbP}{\mathbb{P}}
\newcommand{\bbQ}{\mathbb{Q}}
\newcommand{\bbR}{\mathbb{R}}
\newcommand{\bbS}{\mathbb{S}}
\newcommand{\bbT}{\mathbb{T}}
\newcommand{\bbU}{\mathbb{U}}
\newcommand{\bbV}{\mathbb{V}}
\newcommand{\bbW}{\mathbb{W}}
\newcommand{\bbX}{\mathbb{X}}
\newcommand{\bbY}{\mathbb{Y}}
\newcommand{\bbZ}{\mathbb{Z}}

%% script math capitals
\newcommand{\sA}{\mathscr{A}}
\newcommand{\sB}{\mathscr{B}}
\newcommand{\sC}{\mathscr{C}}
\newcommand{\sD}{\mathscr{D}}
\newcommand{\sE}{\mathscr{E}}
\newcommand{\sF}{\mathscr{F}}
\newcommand{\sG}{\mathscr{G}}
\newcommand{\sH}{\mathscr{H}}
\newcommand{\sI}{\mathscr{I}}
\newcommand{\sJ}{\mathscr{J}}
\newcommand{\sK}{\mathscr{K}}
\newcommand{\sL}{\mathscr{L}}
\newcommand{\sM}{\mathscr{M}}
\newcommand{\sN}{\mathscr{N}}
\newcommand{\sO}{\mathscr{O}}
\newcommand{\sP}{\mathscr{P}}
\newcommand{\sQ}{\mathscr{Q}}
\newcommand{\sR}{\mathscr{R}}
\newcommand{\sS}{\mathscr{S}}
\newcommand{\sT}{\mathscr{T}}
\newcommand{\sU}{\mathscr{U}}
\newcommand{\sV}{\mathscr{V}}
\newcommand{\sW}{\mathscr{W}}
\newcommand{\sX}{\mathscr{X}}
\newcommand{\sY}{\mathscr{Y}}
\newcommand{\sZ}{\mathscr{Z}}

\renewcommand{\phi}{\varphi}
%\renewcommand{\emptyset}{\O}

\providecommand{\abs}[1]{\lvert #1 \rvert}
\providecommand{\norm}[1]{\lVert #1 \rVert}
\providecommand{\x}{\times}
\providecommand{\ar}{\rightarrow}
\providecommand{\arr}{\longrightarrow}


%----------Theorems----------

\newtheorem{theorem}{Theorem}[section]
\newtheorem{proposition}[theorem]{Proposition}
\newtheorem{lemma}[theorem]{Lemma}
\newtheorem{corollary}[theorem]{Corollary}

\theoremstyle{definition}
\newtheorem{definition}[theorem]{Definition}
\newtheorem{nondefinition}[theorem]{Non-Definition}
\newtheorem{exercise}[theorem]{Exercise}

\numberwithin{equation}{subsection}


%----------Title-------------
\title{Homework 1}
\author{Yifan}

\begin{document}

\pagestyle{plain}


%%---  sheet number for theorem counter
%\setcounter{section}{1}

\begin{center}
{\large Homework 1} \\
\vspace{.2in}
Yifan Liu
\end{center}

\bigskip \bigskip

% \section{Chapter 1}

\textbf{Exercise 1.2.2} If $A$ and $B$ are sets, show that $A = B$ if and only if $A \subseteq B$ and $B \subseteq A$.
% If $A$ and $B$ are sets, show that $A = B$ if and only if $A \subseteq B$ and $B \subseteq A$.
\begin{proof}
% First assume that $A = B$. According to the definition of the equality of sets, if $x \in A$, then $x \in B$, and conversely if $x \in B$, then $x \in A$. According to the definition of inclusion, $\forall x \in A: x \in B \implies A \subseteq B$
$\Longrightarrow$ (from left to right): If $A = B$, then by definition of the equality of sets, $\forall x \in A: x \in B \text{ and } \forall x \in B: x \in A \text{, in other words, } A \subseteq B \text{ and } B \subseteq A \text{.}$\\
$\Longleftarrow$ (from right to left): $A \subseteq B \text{ and } B \subseteq A \implies \text{by definition of } \subseteq \text{, } \forall x \in A: x \in B \text{ and } \forall x \in B: x \in A \text{, which is the same as the definition of set equality, in other words,}\\ A = B \text{.}$
\end{proof}

\textbf{Exercise 1.2.3} Suppose that \(A, B,\) and \(C\) are sets. If \(A \subseteq B\) and \(B \subseteq C\), show that \(A \subseteq C\).

\begin{proof}
In order to prove $A \subseteq C$, we need to show that $\forall x \in A: x \in C$. \\
Let $x \in A$. Because $A \subseteq B$, by definition, $x$ is also in $B$, in other words, $x \in B$. Because $x \in B$, and $B \subseteq C$, similarly we have $x \in C$. Hence, $\forall x \in A: x \in C$, which means $A \subseteq C$.
\end{proof}

\textbf{Exercise 1.2.4} Show that if \(A\) is a set, then \(\varnothing \subseteq A\).

\begin{proof}
In order to prove $\varnothing \subseteq A$, we need to show that $\forall x \in \varnothing: x \in A$. Because there are no elements in $\varnothing$, $x \in \varnothing$ would never be true. When the antecedent is false, no matter what the consequent is, the statement is always true. For this reason, $\forall x \in \varnothing: x \in A$. By definition, this is the same as $\varnothing \subseteq A$.
\end{proof}

% \begin{definition}
% We use
% \begin{verbatim}
% \begin{tag}
% Stuff
% \end{tag}
% \end{verbatim}
% environments for various things where tag = definition, lemma, equation, array, theorem.  There are many more.  Look in the header of the .tex file for more.
% \end{definition}

% If we have an equation, then it becomes pulled out of text.
% \begin{equation*}
% x + y = z.
% \end{equation*}

% Use dollar signs for math symbols not in equations $x+y = z$.

% Its easy to add special symbols such as $\alpha, \pm, \sum, x \in A$.

% If you want to find a special symbol look online.





\end{document}