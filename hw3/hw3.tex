\documentclass[12pt]{article}


%----------Packages----------
\usepackage{amsmath}
\usepackage{amssymb}
\usepackage{amsthm}
%\usepackage{amsrefs}
\usepackage{dsfont}
\usepackage{mathrsfs}
\usepackage{stmaryrd}
\usepackage[all]{xy}
\usepackage[mathcal]{eucal}
\usepackage{verbatim}  %%includes comment environment
\usepackage{fullpage}  %%smaller margins
\usepackage{hyperref}
\usepackage{setspace}
\onehalfspacing
%----------Commands----------

%%penalizes orphans
\clubpenalty=9999
\widowpenalty=9999

%% bold math capitals
\newcommand{\bA}{\mathbf{A}}
\newcommand{\bB}{\mathbf{B}}
\newcommand{\bC}{\mathbf{C}}
\newcommand{\bD}{\mathbf{D}}
\newcommand{\bE}{\mathbf{E}}
\newcommand{\bF}{\mathbf{F}}
\newcommand{\bG}{\mathbf{G}}
\newcommand{\bH}{\mathbf{H}}
\newcommand{\bI}{\mathbf{I}}
\newcommand{\bJ}{\mathbf{J}}
\newcommand{\bK}{\mathbf{K}}
\newcommand{\bL}{\mathbf{L}}
\newcommand{\bM}{\mathbf{M}}
\newcommand{\bN}{\mathbf{N}}
\newcommand{\bO}{\mathbf{O}}
\newcommand{\bP}{\mathbf{P}}
\newcommand{\bQ}{\mathbf{Q}}
\newcommand{\bR}{\mathbf{R}}
\newcommand{\bS}{\mathbf{S}}
\newcommand{\bT}{\mathbf{T}}
\newcommand{\bU}{\mathbf{U}}
\newcommand{\bV}{\mathbf{V}}
\newcommand{\bW}{\mathbf{W}}
\newcommand{\bX}{\mathbf{X}}
\newcommand{\bY}{\mathbf{Y}}
\newcommand{\bZ}{\mathbf{Z}}

%% blackboard bold math capitals
\newcommand{\bbA}{\mathbb{A}}
\newcommand{\bbB}{\mathbb{B}}
\newcommand{\bbC}{\mathbb{C}}
\newcommand{\bbD}{\mathbb{D}}
\newcommand{\bbE}{\mathbb{E}}
\newcommand{\bbF}{\mathbb{F}}
\newcommand{\bbG}{\mathbb{G}}
\newcommand{\bbH}{\mathbb{H}}
\newcommand{\bbI}{\mathbb{I}}
\newcommand{\bbJ}{\mathbb{J}}
\newcommand{\bbK}{\mathbb{K}}
\newcommand{\bbL}{\mathbb{L}}
\newcommand{\bbM}{\mathbb{M}}
\newcommand{\bbN}{\mathbb{N}}
\newcommand{\bbO}{\mathbb{O}}
\newcommand{\bbP}{\mathbb{P}}
\newcommand{\bbQ}{\mathbb{Q}}
\newcommand{\bbR}{\mathbb{R}}
\newcommand{\bbS}{\mathbb{S}}
\newcommand{\bbT}{\mathbb{T}}
\newcommand{\bbU}{\mathbb{U}}
\newcommand{\bbV}{\mathbb{V}}
\newcommand{\bbW}{\mathbb{W}}
\newcommand{\bbX}{\mathbb{X}}
\newcommand{\bbY}{\mathbb{Y}}
\newcommand{\bbZ}{\mathbb{Z}}

%% script math capitals
\newcommand{\sA}{\mathscr{A}}
\newcommand{\sB}{\mathscr{B}}
\newcommand{\sC}{\mathscr{C}}
\newcommand{\sD}{\mathscr{D}}
\newcommand{\sE}{\mathscr{E}}
\newcommand{\sF}{\mathscr{F}}
\newcommand{\sG}{\mathscr{G}}
\newcommand{\sH}{\mathscr{H}}
\newcommand{\sI}{\mathscr{I}}
\newcommand{\sJ}{\mathscr{J}}
\newcommand{\sK}{\mathscr{K}}
\newcommand{\sL}{\mathscr{L}}
\newcommand{\sM}{\mathscr{M}}
\newcommand{\sN}{\mathscr{N}}
\newcommand{\sO}{\mathscr{O}}
\newcommand{\sP}{\mathscr{P}}
\newcommand{\sQ}{\mathscr{Q}}
\newcommand{\sR}{\mathscr{R}}
\newcommand{\sS}{\mathscr{S}}
\newcommand{\sT}{\mathscr{T}}
\newcommand{\sU}{\mathscr{U}}
\newcommand{\sV}{\mathscr{V}}
\newcommand{\sW}{\mathscr{W}}
\newcommand{\sX}{\mathscr{X}}
\newcommand{\sY}{\mathscr{Y}}
\newcommand{\sZ}{\mathscr{Z}}

\renewcommand{\phi}{\varphi}
%\renewcommand{\emptyset}{\O}

\providecommand{\abs}[1]{\lvert #1 \rvert}
\providecommand{\norm}[1]{\lVert #1 \rVert}
\providecommand{\x}{\times}
\providecommand{\ar}{\rightarrow}
\providecommand{\arr}{\longrightarrow}


%----------Theorems----------

\newtheorem{theorem}{Theorem}[section]
\newtheorem{proposition}[theorem]{Proposition}
\newtheorem{lemma}[theorem]{Lemma}
\newtheorem{corollary}[theorem]{Corollary}

\theoremstyle{definition}
\newtheorem{definition}[theorem]{Definition}
\newtheorem{nondefinition}[theorem]{Non-Definition}
\newtheorem{exercise}[theorem]{Exercise}

\numberwithin{equation}{subsection}


%----------Title-------------
\title{Homework 3}
\author{Yifan}

\begin{document}

\pagestyle{plain}


%%---  sheet number for theorem counter
%\setcounter{section}{1}

\begin{center}
{\large Homework 3} \\
\vspace{.2in}
Yifan Liu
\end{center}

\bigskip \bigskip

% \section{Chapter 1}

\textbf{Exercise 1.4.9} If \(A_{1}\) has \(k_{1}\) elements, \(A_{2}\) has \(k_{2}\) elements, \(\ldots, A_{n}\) has \(k_{n}\)elements, show that \(\#\left(A_{1} \times A_{2} \times \cdots \times A_{n}\right)=\left(\# A_{1}\right)\left(\# A_{2}\right) \cdots\left(\# A_{n}\right)=\) \(k_{1} k_{2} \cdots k_{n} .\)
\begin{proof}
When $n=1$, apparently $\#A_{1}=\#A_1$.\\
Given \(\#\left(A_{1} \times A_{2} \times \cdots \times A_{n}\right)=\left(\# A_{1}\right)\left(\# A_{2}\right) \cdots\left(\# A_{n}\right)\), then:\\
For any $(a_1,a_2,\cdots,a_n) \in (A_{1} \times A_{2} \times \cdots \times A_{n})$, there are $k_{n+1}$ different $a_{n+1} \in A_{n+1}$ such that $(a_1,a_2,\cdots,a_{n+1}) \in (A_{1} \times A_{2} \times \cdots \times A_{n+1})$. Hence, \(\#\left(A_{1} \times A_{2} \times \cdots \times A_{n+1}\right)
=\left(\#\left(A_{1} \times A_{2} \times \cdots \times A_{n}\right)\right)\left(\#A_{n+1}\right)
=\left(\# A_{1}\right)\left(\# A_{2}\right) \cdots\left(\# A_{n+1}\right)\).
\end{proof}

\textbf{Exercise 2} Show that, for all \(n \in \mathbb{N},\)
$$
1^{2}+2^{2}+\cdots+n^{2}=\frac{n(n+1)(2 n+1)}{6}
$$
\begin{proof}
If $n = 1$:
$1^{2}=\frac{1\times(1+1)\times(2\times 1+1)}{6}=1$. True.\\
If $1^{2}+2^{2}+\cdots+n^{2}=\frac{n(n+1)(2 n+1)}{6}$ holds for $n$, then it also holds for $n + 1$:\\
\begin{align*}
1^{2}+2^{2}+\cdots+n^{2}+\cdots+(n+1)^{2}&=\frac{n(n+1)(2 n+1)}{6}+(n+1)^{2}\\
&=\frac{n(n+1)(2 n+1)+6n^{2}+12n+6}{6}\\
&=\frac{2 n^{3}+9 n^{2}+13 n+6}{6}\\
&=\frac{(n+1)(n+1+1)(2(n+1)+1)}{6}
\end{align*}
Thus, $\forall n \in \bbN$, $1^{2}+2^{2}+\cdots+n^{2}=\frac{n(n+1)(2 n+1)}{6}$.

\end{proof}

\textbf{Exercise 3} Show that, for all \(x \neq 1\) and \(n \in \mathbb{N}\)
$$
1+x+x^{2} \ldots+x^{n}=\frac{x^{n+1}-1}{x-1}
$$
\begin{proof}
Given $x\neq1$\\
If $n = 1$, apparently $1 + x = \frac{x^{1+1}-1}{x-1}= \frac{x^{2}-1}{x-1}$.\\
If the equation holds for $n$, then it also holds for $n+1$:\\
\begin{align*}
1+x+x^{2} \ldots+x^{n}+x^{n+1}
&=\frac{x^{n+1}-1}{x-1}+x^{n+1}\\
&=\frac{x^{n+1}-1+x^{n+2}-x^{n+1}}{x-1}\\
&=\frac{x^{n+2}-1}{x-1}\\
\end{align*}
Hence, $1+x+x^{2} \ldots+x^{n}=\frac{x^{n+1}-1}{x-1}$.
\end{proof}

\textbf{Exercise 1.7.9} Prove Pigeon Hole Principle (Suppose that \(A\) is a set with \(n\) elements, \(B\) is a set with \(m\) elements, and \(n>m\). If \(f: A \rightarrow B\) is a function, there
are at least two distinct elements of \(A\) that correspond to the same element
of \(B\).)
\begin{proof}
Suppose there are no two elements of $A$ that correspond to the same element of $B$, then there should be at least $n$ elements in $B$, while there is actually only $m$ elements ($n>m$). Contradiction.\\
Hence, there are at least two elements of $A$ that correspond to the same element of $B$.
\end{proof}

\textbf{Exercise 1.7.15(i)}\(f: \mathbb{N} \rightarrow \mathbb{N}, f(n)=2 n\)
\begin{proof}
Injective:
If \(f(a)=f\left(a^{\prime}\right),\) then \(2a=2a^{\prime}\), then \(a=a^{\prime}\).\\
Not surjective:
$\exists 1 \in \bbN \text{ s.t. } \nexists n \in \bbN: 2n=1.
$ In other words, $1 \notin f(\bbN)$.
So $f(\bbN) \neq \bbN$.
\end{proof}

\textbf{Exercise 1.7.15(iii)}\(f: \mathbb{N} \rightarrow \mathbb{Q}, f(n)=n\)
\begin{proof}
Injective:
If \(f(a)=f\left(a^{\prime}\right),\) then \(a=a^{\prime}\).\\
Not surjective:
$\exists 0.1 \in \bbQ \text{ s.t. } \nexists n \in \bbN: n=0.1.
$ In other words, $0.1 \notin f(\bbN)$.
So $f(\bbN) \neq \bbQ$
\end{proof}

\textbf{Exercise 1.7.15(v)} Write the real numbers in terms of their decimal expansions. As
usual, we do not allow a real number to end in all 9's repeating.
Let \(f: \mathbb{R} \rightarrow \mathbb{N}\) be defined by: \(f(x)\) equals the third digit of \(x\) after
the decimal point (this is called the Michelle function).
\begin{proof}
We have to suppose $0 \in \bbN$ or the definition wouldn't make sense.\\
Not injective:
$1.001 \neq 2.001$ and $f(1.001)=f(2.001)=1$.\\
Not surjective:
$10 \in \bbN$ but $10 \notin f(\bbR)$ as $f(x)$ is always one single DIGIT.
\end{proof}

\textbf{Exercise 1.7.20} Given \(f: A \rightarrow B,\) suppose there exist \(g, h: B \rightarrow A\)
so that \(f \circ g=I_{B}\) and \(h \circ f=I_{A} .\) Show that \(f\) is a bijection and that
\(g=h=f^{-1} .\)
\begin{proof}
First, \(f\) is surjective. For any \(b \in B\), \(\exists a=g(b) \) s.t. \(f(a)=\)
\(f\left(g(b)\right)=f \circ g(b)=I_{B}(b)=b .\) So \(f\) is surjective.\\
$f$ is also injective. If $a_1, a_2 \in A $ and $f(a_1)=f(a_2)$, then
\begin{align*}
a_{1} &=I_{A}\left(a_{1}\right) \\
&=h \circ f\left(a_{1}\right) \\
&=h\left(f\left(a_{1}\right)\right) \\
&=h\left(f\left(a_{2}\right)\right) \\
&=h \circ f\left(a_{2}\right) \\
&=I_{A}\left(a_{2}\right) \\
&=a_2
\end{align*}
Hence, $f$ is bijective. \\
Then we prove \(g=h=f^{-1} .\)\\
(Note: Definition of inverse: If \(b \in B,\) then we set
\(f^{-1}(b)=a\) where \(a\) is the unique element of \(A\) such that \(f(a)=b\).
)\\
Let $b \in B$, then there exists a unique $a$ such that $f(a)=b$.\\ $h(b)=h(f(a))=I_A(a)=a$, so $h=f^{-1}$.
\\$f(g(b))=I_B(b)=b=f(a)$, and because $f$ is bijective, $g(b)=a$, so $g=f^{-1}$.\\
Hence, $g=h=f^{-1}$.
\end{proof}

\textbf{Exercise 1.7.24(i)} \(f\left(A_{1} \cup A_{2}\right)=f\left(A_{1}\right) \cup f\left(A_{2}\right)\)
\begin{proof} First,
\(f\left(A_{1} \cup A_{2}\right) \subseteq f\left(A_{1}\right) \cup f\left(A_{2}\right)\):\\
\(\forall y \in f\left(A_{1} \cup A_{2}\right),\) \(\exists x \in A_{1} \cup A_{2}\)
such that \(f(x)=y .\) By the definition of union, \(x \in A_{1}\) or (``or'' in mathematics) \(x \in A_{2} .\) This implies that \(y \in f\left(A_{1}\right)\)
or \(y \in f\left(A_{2}\right) .\) Therefore, \(y \in f\left(A_{1}\right) \cup f\left(A_{2}\right) .\)\\
Second, \(f\left(A_{1}\right) \cup f\left(A_{2}\right) \subseteq f\left(A_{1} \cup A_{2}\right)\):\\
Let \(y \in f\left(A_{1}\right) \cup f\left(A_{2}\right) .\) Then \(y \in f\left(A_{1}\right)\) or \(y \in f\left(A_{2}\right)\) so there
is \(x_{1} \in A_{1}\) such that \(f\left(x_{1}\right)=y\) or there is \(x_{2} \in A_{2}\) such that \(f\left(x_{2}\right)=y .\) In either case, \(\exists x \in A_{1} \cup A_{2}\) such that \(f(x)=y .\) Therefore \(f\left(A_{1}\right) \cup f\left(A_{2}\right) \subseteq f\left(A_{1} \cup A_{2}\right)\).\\
Hence, \(f\left(A_{1} \cup A_{2}\right)=f\left(A_{1}\right) \cup f\left(A_{2}\right) .\)
\end{proof}

\textbf{Exercise 1.7.24(iv)} \(f^{-1}\left(B_{1} \cap B_{2}\right)=f^{-1}\left(B_{1}\right) \cap f^{-1}\left(B_{2}\right)\)
\begin{proof}
First, \(f^{-1}\left(B_{1} \cap B_{2}\right) \subseteq f^{-1}\left(B_{1}\right) \cap f^{-1}\left(B_{2}\right)\):\\
Take \(x \in f^{-1}\left(B_{1} \cap B_{2}\right) .\) Then \(f(x) \in B_{1} \cap B_{2}\) so \(f(x) \in B_{1}\) and \(f(x) \in B_{2}\).
Hence \(x \in f^{-1}\left(B_{1}\right)\) and \(x \in f^{-1}\left(B_{2}\right) .\) That is, \(x \in f^{-1}\left(B_{1}\right) \cap f^{-1}\left(B_{2}\right)\) and so
\(f^{-1}\left(B_{1} \cap B_{2}\right) \subseteq f^{-1}\left(B_{1}\right) \cap f^{-1}\left(B_{2}\right)\).\\
Second, \(f^{-1}\left(B_{1}\right) \cap f^{-1}\left(B_{2}\right) \subseteq f^{-1}\left(B_{1} \cap B_{2}\right)\):\\
Take \(x \in f^{-1}\left(B_{1}\right) \cap f^{-1}\left(B_{2}\right) .\) Then \(x \in f^{-1}\left(B_{1}\right)\) and \(x \in f^{-1}\left(B_{2}\right)\). Then \(f(x) \in B_{1}\) and \(f(x) \in B_{2}\) , so \(f(x) \in B_{1} \cap B_{2}\). Hence \(x \in f^{-1}\left(B_{1} \cap B_{2}\right)\). That is, \(f^{-1}\left(B_{1}\right) \cap f^{-1}\left(B_{2}\right) \subseteq f^{-1}\left(B_{1} \cap B_{2}\right)\).\\
Therefore, \(f^{-1}\left(B_{1} \cap B_{2}\right)=f^{-1}\left(B_{1}\right) \cap f^{-1}\left(B_{2}\right)\).
\end{proof}

\textbf{Exercise 1.7.25} Find an example to show that equality does not necessarily hold in (ii) \(f\left(A_{1} \cap A_{2}\right) \subseteq f\left(A_{1}\right) \cap f\left(A_{2}\right)\)
\begin{proof}
Example:\\
$$
A_1=\{0,1\}, A_2=\{0,2\}, f(0)=0, f(1)=f(2)=1
$$
Then
$$
f\left(A_{1} \cap A_{2}\right)=f(\{0\})=\{0\},\;
f\left(A_{1}\right) \cap f\left(A_{2}\right)=\{0,1\}\cap\{0,1\}=\{0,1\}, \;
$$
$$f\left(A_{1} \cap A_{2}\right) \neq f\left(A_{1}\right) \cap f\left(A_{2}\right)$$
\end{proof}





% \begin{definition}
% We use
% \begin{verbatim}
% \begin{tag}
% Stuff
% \end{tag}
% \end{verbatim}
% environments for various things where tag = definition, lemma, equation, array, theorem.  There are many more.  Look in the header of the .tex file for more.
% \end{definition}

% If we have an equation, then it becomes pulled out of text.
% \begin{equation*}
% x + y = z.
% \end{equation*}

% Use dollar signs for math symbols not in equations $x+y = z$.

% Its easy to add special symbols such as $\alpha, \pm, \sum, x \in A$.

% If you want to find a special symbol look online.





\end{document}