\documentclass[12pt]{article}


%----------Packages----------
\usepackage{amsmath}
\usepackage{amssymb}
\usepackage{amsthm}
%\usepackage{amsrefs}
\usepackage{dsfont}
\usepackage{mathrsfs}
\usepackage{stmaryrd}
\usepackage[all]{xy}
\usepackage[mathcal]{eucal}
\usepackage{verbatim}  %%includes comment environment
\usepackage{fullpage}  %%smaller margins
\usepackage{hyperref}
\usepackage{setspace}
\onehalfspacing
%----------Commands----------

%%penalizes orphans
\clubpenalty=9999
\widowpenalty=9999

%% bold math capitals
\newcommand{\bA}{\mathbf{A}}
\newcommand{\bB}{\mathbf{B}}
\newcommand{\bC}{\mathbf{C}}
\newcommand{\bD}{\mathbf{D}}
\newcommand{\bE}{\mathbf{E}}
\newcommand{\bF}{\mathbf{F}}
\newcommand{\bG}{\mathbf{G}}
\newcommand{\bH}{\mathbf{H}}
\newcommand{\bI}{\mathbf{I}}
\newcommand{\bJ}{\mathbf{J}}
\newcommand{\bK}{\mathbf{K}}
\newcommand{\bL}{\mathbf{L}}
\newcommand{\bM}{\mathbf{M}}
\newcommand{\bN}{\mathbf{N}}
\newcommand{\bO}{\mathbf{O}}
\newcommand{\bP}{\mathbf{P}}
\newcommand{\bQ}{\mathbf{Q}}
\newcommand{\bR}{\mathbf{R}}
\newcommand{\bS}{\mathbf{S}}
\newcommand{\bT}{\mathbf{T}}
\newcommand{\bU}{\mathbf{U}}
\newcommand{\bV}{\mathbf{V}}
\newcommand{\bW}{\mathbf{W}}
\newcommand{\bX}{\mathbf{X}}
\newcommand{\bY}{\mathbf{Y}}
\newcommand{\bZ}{\mathbf{Z}}

%% blackboard bold math capitals
\newcommand{\bbA}{\mathbb{A}}
\newcommand{\bbB}{\mathbb{B}}
\newcommand{\bbC}{\mathbb{C}}
\newcommand{\bbD}{\mathbb{D}}
\newcommand{\bbE}{\mathbb{E}}
\newcommand{\bbF}{\mathbb{F}}
\newcommand{\bbG}{\mathbb{G}}
\newcommand{\bbH}{\mathbb{H}}
\newcommand{\bbI}{\mathbb{I}}
\newcommand{\bbJ}{\mathbb{J}}
\newcommand{\bbK}{\mathbb{K}}
\newcommand{\bbL}{\mathbb{L}}
\newcommand{\bbM}{\mathbb{M}}
\newcommand{\bbN}{\mathbb{N}}
\newcommand{\bbO}{\mathbb{O}}
\newcommand{\bbP}{\mathbb{P}}
\newcommand{\bbQ}{\mathbb{Q}}
\newcommand{\bbR}{\mathbb{R}}
\newcommand{\bbS}{\mathbb{S}}
\newcommand{\bbT}{\mathbb{T}}
\newcommand{\bbU}{\mathbb{U}}
\newcommand{\bbV}{\mathbb{V}}
\newcommand{\bbW}{\mathbb{W}}
\newcommand{\bbX}{\mathbb{X}}
\newcommand{\bbY}{\mathbb{Y}}
\newcommand{\bbZ}{\mathbb{Z}}

%% script math capitals
\newcommand{\sA}{\mathscr{A}}
\newcommand{\sB}{\mathscr{B}}
\newcommand{\sC}{\mathscr{C}}
\newcommand{\sD}{\mathscr{D}}
\newcommand{\sE}{\mathscr{E}}
\newcommand{\sF}{\mathscr{F}}
\newcommand{\sG}{\mathscr{G}}
\newcommand{\sH}{\mathscr{H}}
\newcommand{\sI}{\mathscr{I}}
\newcommand{\sJ}{\mathscr{J}}
\newcommand{\sK}{\mathscr{K}}
\newcommand{\sL}{\mathscr{L}}
\newcommand{\sM}{\mathscr{M}}
\newcommand{\sN}{\mathscr{N}}
\newcommand{\sO}{\mathscr{O}}
\newcommand{\sP}{\mathscr{P}}
\newcommand{\sQ}{\mathscr{Q}}
\newcommand{\sR}{\mathscr{R}}
\newcommand{\sS}{\mathscr{S}}
\newcommand{\sT}{\mathscr{T}}
\newcommand{\sU}{\mathscr{U}}
\newcommand{\sV}{\mathscr{V}}
\newcommand{\sW}{\mathscr{W}}
\newcommand{\sX}{\mathscr{X}}
\newcommand{\sY}{\mathscr{Y}}
\newcommand{\sZ}{\mathscr{Z}}

\renewcommand{\phi}{\varphi}
%\renewcommand{\emptyset}{\O}

\providecommand{\abs}[1]{\lvert #1 \rvert}
\providecommand{\norm}[1]{\lVert #1 \rVert}
\providecommand{\x}{\times}
\providecommand{\ar}{\rightarrow}
\providecommand{\arr}{\longrightarrow}


%----------Theorems----------

\newtheorem{theorem}{Theorem}[section]
\newtheorem{proposition}[theorem]{Proposition}
\newtheorem{lemma}[theorem]{Lemma}
\newtheorem{corollary}[theorem]{Corollary}

\theoremstyle{definition}
\newtheorem{definition}[theorem]{Definition}
\newtheorem{nondefinition}[theorem]{Non-Definition}
\newtheorem{exercise}[theorem]{Exercise}

\numberwithin{equation}{subsection}


%----------Title-------------
\title{Homework 7}
\author{Yifan}

\begin{document}

\pagestyle{plain}


%%---  sheet number for theorem counter
%\setcounter{section}{1}

\begin{center}
{\large Homework 7} \\
\vspace{.2in}
Yifan Liu
\end{center}

\bigskip \bigskip

% \section{Chapter 1}


\textbf{Exercise 1} If $\left(x_{n}\right) \rightarrow x$ and $\left(y_{n}\right) \rightarrow y$ prove that $\left(x_{n}+y_{n}\right) \rightarrow x+y .$


We prove more generally that $\left(x_{n}\pm y_{n}\right) \rightarrow x\pm y $ so that it can be used in question 3. 
\begin{proof}
    Given $\varepsilon > 0$. \\
    For $\frac{\varepsilon}{2}$, $\exists N_1 \in \bbN \text{ s.t. } \forall n > N_1, |x_n - x| < \frac{\varepsilon}{2}$. \\
    For $\frac{\varepsilon}{2}$, $\exists N_2 \in \bbN \text{ s.t. } \forall n > N_2, |y_n - y| < \frac{\varepsilon}{2}$. \\
    Let $n = \max\{N_1, N_2\}$. Then $\forall n > N, |(x_n\pm y_n)-(x\pm y)| = |(x_n - x) \pm  (y_n - y) | \leq |x_n-x| + |y_n -y| < \frac{\varepsilon}{2} + \frac{\varepsilon}{2} = \varepsilon. $ \\
    Hence, $\left(x_{n}\pm y_{n}\right) \rightarrow x\pm y .$
\end{proof}


\textbf{Exercise 2} If $\left(x_{n}\right) \rightarrow x$ and $x_{n} \geq 0$ for every $n \in \mathbb{N}$ prove that $x \geq 0$.
\begin{proof}
    Suppose $x<0$, then let $\varepsilon = \frac{|x|}{2}$. \\
    Because $\left(x_{n}\right) \rightarrow x$, for this $\varepsilon$, $\exists N \in \bbN$ s.t. $\forall n > N, |x_n - x| < \varepsilon = \frac{|x|}{2}.$\\
    $x_n \leq |x_n -x| +x < \frac{|x|}{2}+x = -\frac{x}{2}+x = \frac{x}{2} < 0$, which is contradictory with the assumption that $x_{n} \geq 0$.\\
    Therefore, $x \geq 0$.
\end{proof}


\textbf{Exercise 3} If $\left(x_{n}\right) \rightarrow x,\left(y_{n}\right) \rightarrow y$ and $x_{n} \leq y_{n}$ for every $n \in \mathbb{N},$ show that $x \leq y .$
\begin{proof}
    % Suppose $x>y$, then let $\varepsilon = \frac{|x-y|}{2}$. \\
    Consider $(z_n) = (y_n - x_n). $ \\
    According to Exercise 1 (strengthened version), $\left(z_{n}\right) \rightarrow y-x$. \\
    Also, $x_{n} \leq y_{n},$  so $z_n \geq 0$ for every $n \in \mathbb{N}$.\\
    Then, according to Exercise 2, we have $y-x\geq 0$, which implies that $x\leq y$. 
\end{proof}


\textbf{Exercise 4} Let $\left(x_{n}\right) \rightarrow w,\left(z_{n}\right) \rightarrow w$ and suppose $\left(y_{n}\right)$ is such that $x_{n} \leq y_{n} \leq z_{n} .$ Prove that $\left(y_{n}\right) \rightarrow w .$
\begin{proof}
    Because $\left(x_{n}\right) \rightarrow w,\left(z_{n}\right) \rightarrow w$, so for any $\varepsilon > 0$: \\
    $\exists N_{1}: \forall n>N_{1}:\left|x_{n}-w\right|<\varepsilon$\\
    $\exists N_{2}: \forall n>N_{2}:\left|z_{n}-w\right|<\varepsilon$\\
    Let $N = \max\{N_1, N_2\}$, so $\forall n > N :$\\
    $w-\varepsilon < x_n < w+\varepsilon$.\\
    $w-\varepsilon < z_n < w+\varepsilon$.\\
    Because $x_{n} \leq y_{n}$,  $w-\varepsilon < y_n $; 
    because $y_{n} \leq z_{n}$,  $y_n < w +\varepsilon$.\\
    Therefore, 
    $w-\varepsilon < y_n < w+\varepsilon$, which means $\left|y_{n}-w\right|<\varepsilon$.\\
    Since this holds for any $\varepsilon$, we can conclude that $\left(y_{n}\right) \rightarrow w .$
    


\end{proof}


\textbf{Definition} Let $(x_n)$ be a sequence. For every $m \in N$ we define the $m$-tail of $(x_n )$ as the sequence $(x_{n+m} ) = (x_{m+1} , x_{m+2}, . . .).$
For example, the $3$-tail of $(2, 4, 6, 8, 10, . . .)$ is $(8, 10, 12, . . .)$.
This notion is useful to capture the ``end behaviour'' of the
terms of a sequence.

\bigskip


\textbf{Exercise 5} Prove that $(x_n)$ converges if and only its $m$-tail $(x_{n+m} )$ converges. Moreover, show that if this is the case then they both
converge to the same limit. Think about how this ``strengthens''
many of our convergence results (e.g. Questions 3 and 5(do you mean 4?)).
\begin{proof}
    Let $(y_n)$ be the $m$-tail of $(x_n)$, that is, $y_n = x_{n+m}$.

    Left to Right:

    Given $\left(x_{n}\right) \rightarrow x$, which means $\forall \varepsilon > 0: \exists N \in \bbN: \forall n > N: |x_n-x| < \varepsilon$.
    
    Then $\forall \varepsilon > 0: \exists N' = \max\{1, N - m\}: \forall n' > N': |y_{n'}-x| = |x_{n'+m} - x| < \varepsilon$ where $n'+m > N'+m \geq N$.

    Therefore, $(y_n)$, or $(x_{n+m})$,  converges to $x$. 

    \bigskip
    Right to Left:

    Given $\left(y_{n}\right) \rightarrow y$, which means $\forall \varepsilon > 0: \exists N \in \bbN: \forall n > N: |y_n-y| < \varepsilon$.
    
    Then $\forall \varepsilon > 0: \exists N' = N+m: \forall n' > N': |x_{n'}-y| = |y_{n'-m} - y| < \varepsilon$ where $n'-m > N'-m = N$.

    Therefore, $(x_n)$  converges to $y$. 



\end{proof}


Exercise 3 can be modified in this way:

If $\left(x_{n}\right) \rightarrow x,\left(y_{n}\right) \rightarrow y$ and $x_{n} \leq y_{n}$ for all $n\geq N$ where $N \in \mathbb{N},$ then $x \leq y .$

\bigskip

Exercise 4 can be modified in this way:

Let $\left(x_{n}\right) \rightarrow w,\left(z_{n}\right) \rightarrow w$ and suppose $\left(y_{n}\right)$ is such that for all $n\geq N$ where $N \in \mathbb{N},$ $x_{n} \leq y_{n} \leq z_{n} .$ In this case, $\left(y_{n}\right) \rightarrow w .$

\bigskip

\textbf{Exercise 6} Let $\left(y_{n}\right)$ be defined inductively by $y_{1}:=1, y_{n+1}:=\frac{1}{4}\left(2 y_{n}+3\right)$.
Use the Monotone Convergence Theorem and the previous question to show that $\left(y_{n}\right) \rightarrow \frac{3}{2}$.
\begin{proof}
    First, we prove $y_n < \frac{3}{2}$ for all $n \in \bbN$. \\
    1. $y_1 = 1 < \frac{3}{2}$.\\
    2. If $y_n < \frac{3}{2}$, then $y_{n+1} < \frac{1}{4}(2\cdot\frac{3}{2}+3) = \frac{3}{2}$. \\
    Then by induction, we know $y_n < \frac{3}{2}$ for all $n \in \bbN$, so $(y_n)$ is bounded above by $\frac{3}{2}$. 

    \bigskip
    Then we prove its monotone. \\
    For $n \in \bbN$: 
    $y_{n+1} - y_n = \frac{1}{4}\left(2 y_{n}+3\right) - y_n = -\frac{1}{2}y_n + \frac{3}{4} > -\frac{1}{2}\cdot \frac{3}{2} + \frac{3}{4} = 0$.
    Hence, $(y_n)$ is increasing. 

    \bigskip

    Therefore, the limit of $(y_n)$ exists. Solve for $y=\frac{1}{4}\left(2 y+3\right)$, we get $y =\frac{3}{2}$, so $\left(y_{n}\right) \rightarrow \frac{3}{2}$.

\end{proof}


\textbf{Definition} A sequence $\left(x_{n}\right)$ is said to be Cauchy if for every $\varepsilon>0$ there is a natural number $H \in \mathbb{N}$ such that for all
$n, m \geq H,$ the terms $x_{n}$ and $x_{m}$ satisfy $\left|x_{n}-x_{m}\right|<\varepsilon .$
\bigskip

\textbf{Exercise 7} If $(x_n )$ is a convergent sequence, prove that it is Cauchy.

\begin{proof}
    For any $\varepsilon > 0$:\\
    Since $(x_n)$ is convergent to some $x$, consider $\frac{\varepsilon}{2}$, then there exists $N \in \bbN$ s.t. $\forall n > N: |x_n - x| < \frac{\varepsilon}{2}$. Then for all $n, m \geq N,$ $\left|x_{n}-x_{m}\right| = |(x_n - x) - (x_m - x)|\leq |x_n - x| + |x_m - x| < \frac{\varepsilon}{2}+ \frac{\varepsilon}{2} = \varepsilon .$ So the sequence is Cauchy. 
\end{proof}

\textbf{Exercise 8} Show that, if $\left(a_{n}\right)_{n \in \mathbb{N}}$ and $\left(b_{n}\right)_{n \in \mathbb{N}}$ are Cauchy sequences
in $\mathbb{R},$ then $\left(a_{n}+b_{n}\right)_{n \in \mathbb{N}}$ and $\left(a_{n} \cdot b_{n}\right)_{n \in \mathbb{N}}$ are Cauchy sequences in $\mathbb{R}$.
\begin{proof}
    Sum:\\
    Given $\varepsilon > 0$.\\
    Since $\left(a_{n}\right)$ is Cauchy, for $\frac{\varepsilon}{2}>0$ there exists an $N_{1} \in \mathbb{N}$ s.t. if $n,m \geq N_{1}$ then $\left|a_{m}-a_{n}\right|<\frac{\varepsilon}{2}$.\\
    Since $\left(b_{n}\right)$ is Cauchy, for $\frac{\varepsilon}{2}>0$ there exists an $N_{2} \in \mathbb{N}$ s.t. if $n,m \geq N_{2}$ then $\left|b_{m}-b_{n}\right|<\frac{\varepsilon}{2}$.\\
    Let $N = \max\{N_1, N_2\}$. If $n,m \geq N$ then $|(a_m + b_m) - (a_n + b_n)| \leq |a_m - a_n| + |b_m - b_n| < \varepsilon$. So $(a_n+ b_n) $ is Cauchy. 



    \bigskip
    Product:\\ 
    Given $\varepsilon > 0$.\\
    Let $\varepsilon' = \varepsilon / 2M$.\\
    Since $\left(a_{n}\right)$ is Cauchy, for $\varepsilon'>0$ there exists an $N_{1} \in \mathbb{N}$ s.t. if $n,m \geq N_{1}$ then $\left|a_{m}-a_{n}\right|<\varepsilon'$.\\
    Since $\left(b_{n}\right)$ is Cauchy, for $\varepsilon'>0$ there exists an $N_{2} \in \mathbb{N}$ s.t. if $n,m \geq N_{2}$ then $\left|b_{m}-b_{n}\right|<\varepsilon'$.

    Because $(a_n)$ and $(b_n)$ are both Cauchy, $(|a_n|)$ and $(|b_n|)$ are apparently also Cauchy (because $||x_m| - |x_n|| \leq |x_m - x_n| < \varepsilon$), and they are bounded above (take the maximum of the first $N$ terms and $x_N+\varepsilon$, where $\varepsilon$ is any number $>0$ and $N$ is determined by the definition of convergence). Let $M$ be the larger one of the two supremums, then $(|a_n|)$ and $(|b_n|)$ are bounded above by $M$. 

    Let $N = \max\{N_1, N_2\}$. 
    
    If $n,m \geq N$ then $|a_{n} b_{n}-a_{m} b_{m}|=|\left(a_{n} b_{n}-a_{n} b_{m}\right)+\left(a_{n} b_{m}-a_{m} b_{m}\right)|\leq |a_n||b_n-b_m|+|b_m||a_n-a_m| < M\varepsilon' + M\varepsilon' = 2M\varepsilon' = \varepsilon$. So $(a_n\cdot b_n) $ is Cauchy. 

\end{proof}


\end{document}
$\forall \varepsilon>0\left(\exists N \in \mathbb{N}\left(\forall n \in \mathbb{N}\left(n \geq N \Longrightarrow\left|x_{n}-x\right|<\varepsilon\right)\right)\right)$