\documentclass[12pt]{article}


%----------Packages----------
\usepackage{amsmath}
\usepackage{amssymb}
\usepackage{amsthm}
%\usepackage{amsrefs}
\usepackage{dsfont}
\usepackage{mathrsfs}
\usepackage{stmaryrd}
\usepackage[all]{xy}
\usepackage[mathcal]{eucal}
\usepackage{verbatim}  %%includes comment environment
\usepackage{fullpage}  %%smaller margins
\usepackage{hyperref}
\usepackage{setspace}
\onehalfspacing
%----------Commands----------

%%penalizes orphans
\clubpenalty=9999
\widowpenalty=9999

%% bold math capitals
\newcommand{\bA}{\mathbf{A}}
\newcommand{\bB}{\mathbf{B}}
\newcommand{\bC}{\mathbf{C}}
\newcommand{\bD}{\mathbf{D}}
\newcommand{\bE}{\mathbf{E}}
\newcommand{\bF}{\mathbf{F}}
\newcommand{\bG}{\mathbf{G}}
\newcommand{\bH}{\mathbf{H}}
\newcommand{\bI}{\mathbf{I}}
\newcommand{\bJ}{\mathbf{J}}
\newcommand{\bK}{\mathbf{K}}
\newcommand{\bL}{\mathbf{L}}
\newcommand{\bM}{\mathbf{M}}
\newcommand{\bN}{\mathbf{N}}
\newcommand{\bO}{\mathbf{O}}
\newcommand{\bP}{\mathbf{P}}
\newcommand{\bQ}{\mathbf{Q}}
\newcommand{\bR}{\mathbf{R}}
\newcommand{\bS}{\mathbf{S}}
\newcommand{\bT}{\mathbf{T}}
\newcommand{\bU}{\mathbf{U}}
\newcommand{\bV}{\mathbf{V}}
\newcommand{\bW}{\mathbf{W}}
\newcommand{\bX}{\mathbf{X}}
\newcommand{\bY}{\mathbf{Y}}
\newcommand{\bZ}{\mathbf{Z}}

%% blackboard bold math capitals
\newcommand{\bbA}{\mathbb{A}}
\newcommand{\bbB}{\mathbb{B}}
\newcommand{\bbC}{\mathbb{C}}
\newcommand{\bbD}{\mathbb{D}}
\newcommand{\bbE}{\mathbb{E}}
\newcommand{\bbF}{\mathbb{F}}
\newcommand{\bbG}{\mathbb{G}}
\newcommand{\bbH}{\mathbb{H}}
\newcommand{\bbI}{\mathbb{I}}
\newcommand{\bbJ}{\mathbb{J}}
\newcommand{\bbK}{\mathbb{K}}
\newcommand{\bbL}{\mathbb{L}}
\newcommand{\bbM}{\mathbb{M}}
\newcommand{\bbN}{\mathbb{N}}
\newcommand{\bbO}{\mathbb{O}}
\newcommand{\bbP}{\mathbb{P}}
\newcommand{\bbQ}{\mathbb{Q}}
\newcommand{\bbR}{\mathbb{R}}
\newcommand{\bbS}{\mathbb{S}}
\newcommand{\bbT}{\mathbb{T}}
\newcommand{\bbU}{\mathbb{U}}
\newcommand{\bbV}{\mathbb{V}}
\newcommand{\bbW}{\mathbb{W}}
\newcommand{\bbX}{\mathbb{X}}
\newcommand{\bbY}{\mathbb{Y}}
\newcommand{\bbZ}{\mathbb{Z}}

%% script math capitals
\newcommand{\sA}{\mathscr{A}}
\newcommand{\sB}{\mathscr{B}}
\newcommand{\sC}{\mathscr{C}}
\newcommand{\sD}{\mathscr{D}}
\newcommand{\sE}{\mathscr{E}}
\newcommand{\sF}{\mathscr{F}}
\newcommand{\sG}{\mathscr{G}}
\newcommand{\sH}{\mathscr{H}}
\newcommand{\sI}{\mathscr{I}}
\newcommand{\sJ}{\mathscr{J}}
\newcommand{\sK}{\mathscr{K}}
\newcommand{\sL}{\mathscr{L}}
\newcommand{\sM}{\mathscr{M}}
\newcommand{\sN}{\mathscr{N}}
\newcommand{\sO}{\mathscr{O}}
\newcommand{\sP}{\mathscr{P}}
\newcommand{\sQ}{\mathscr{Q}}
\newcommand{\sR}{\mathscr{R}}
\newcommand{\sS}{\mathscr{S}}
\newcommand{\sT}{\mathscr{T}}
\newcommand{\sU}{\mathscr{U}}
\newcommand{\sV}{\mathscr{V}}
\newcommand{\sW}{\mathscr{W}}
\newcommand{\sX}{\mathscr{X}}
\newcommand{\sY}{\mathscr{Y}}
\newcommand{\sZ}{\mathscr{Z}}

\renewcommand{\phi}{\varphi}
%\renewcommand{\emptyset}{\O}

\providecommand{\abs}[1]{\lvert #1 \rvert}
\providecommand{\norm}[1]{\lVert #1 \rVert}
\providecommand{\x}{\times}
\providecommand{\ar}{\rightarrow}
\providecommand{\arr}{\longrightarrow}


%----------Theorems----------

\newtheorem{theorem}{Theorem}[section]
\newtheorem{proposition}[theorem]{Proposition}
\newtheorem{lemma}[theorem]{Lemma}
\newtheorem{corollary}[theorem]{Corollary}

\theoremstyle{definition}
\newtheorem{definition}[theorem]{Definition}
\newtheorem{nondefinition}[theorem]{Non-Definition}
\newtheorem{exercise}[theorem]{Exercise}

\numberwithin{equation}{subsection}


%----------Title-------------
\title{Homework 2}
\author{Yifan}

\begin{document}

\pagestyle{plain}


%%---  sheet number for theorem counter
%\setcounter{section}{1}

\begin{center}
{\large Homework 2} \\
\vspace{.2in}
Yifan Liu
\end{center}

\bigskip \bigskip

% \section{Chapter 1}

\textbf{Exercise 1.3.9(ii)} \(A \cap(B \cap C)=(A \cap B) \cap C \quad(\text {Associative law for intersection})\)
\begin{proof}
$x \in A \cap(B \cap C)$ \\$\iff$ $x \in A \text{ and } x \in (B \cap C)$ \\$\iff$ $x \in A \text{ and } x \in B \text{ and } x \in C$ \\$\iff$ $x \in (A \cap B) \text{ and } x \in C$ \\$\iff$ $x \in (A \cap B) \cap C$.
\end{proof}

\textbf{Exercise 1.3.9(ix)} \(A \triangle B=\varnothing\) iff \(A=B\)
\begin{proof}
\(A \triangle B=\varnothing\) \\$\iff$ \((A \setminus B) \cup(B \setminus A) = \varnothing\) \\$\iff$ $A \setminus B = \varnothing$ and $B \setminus A = \varnothing$ (Note: For all sets $X, Y: X \neq \varnothing \implies \exists x \in X \implies x \in X \cup Y \implies X \cup Y \neq \varnothing$. So here two sets must both be empty. The other direction is trivious: $\varnothing \cup \varnothing = \varnothing$)
\\$\iff$ \(A \subseteq B\) and \(B \subseteq A\) (This is because, $A \setminus B = \varnothing \iff \nexists x \in A \text{ s.t. } x \notin B \iff \forall x \in A: x \in B \iff A \subseteq B$. )
\\$\iff$ \(A=B\)
\end{proof}

\textbf{Exercise 1.3.9(x)} \(A \cap(B \triangle C)=(A \cap B) \triangle(A \cap C) \quad(\) Distributive law of intersection over symmetric difference)
\\\textbf{Lemma}. Intersection distributes over difference
\begin{proof}
\(x \in A \cap (B \setminus C)
\\\iff x \in A \text{ and } x \in B \text{ and } x \notin C
\\\iff (x \in A \cap B) \text{ and } (x \notin A \cap C) \text{ \quad(This is because $x \notin C \implies x \notin A \cap C$)}
\\\iff x \in ((A \cap B) \setminus (A \cap C))
\)
\end{proof}
Then we prove 1.3.9(x)\\
\begin{proof}
(We have known that intersection is distributive over union from Example 1.3.7)
$\\ x \in A \cap(B \triangle C)$
$\\\iff x \in A \cap ((B \setminus C) \cup(C \setminus B)) $
$\\\iff x \in (A \cap (B \setminus C)) \cup (A \cap (C \setminus B))$ \quad(intersection distributive over union)
\newcommand{\m}{(A \cap B)}
\newcommand{\n}{(A \cap C)}
$\\\iff x \in (\m \setminus \n) \cup (\n \setminus \m)$ \quad (lemma)
$\\\iff x \in \m \triangle \n$
\let\m\undefined
\let\n\undefined
\end{proof}

\textbf{Exercise 1.3.9(xii)} \((A \cap B)^{c} = A^{c} \cup B^{c} \quad(\) DeMorgan's Law II)
\begin{proof}
Take $x \in X$. Then $\\ x \in (A \cap B)^{c}$
$\\\iff x \in X \setminus (A \cap B)$
$\\\iff x \in X \text{ and } x \notin A \cap B$
$\\\iff (x \in X \text{ and } x \notin A) \text{ or } (x \in X \text{ and } x \notin B)$
$\\\iff x \in A^{c} \text{ or } x \in B^{c}$
$\\\iff x \in A^{c} \cup B^{c}$
\end{proof}

\textbf{Exercise 1.4.5} Prove that \(A \times \varnothing=\varnothing \times A=\varnothing\)
\begin{proof}
if $\exists (a, b) \in A \times \varnothing$, then \(a \in A\) and \(b \in \varnothing\), while the latter one is impossible. So there are no elements in $A \times \varnothing$, in other words, \(A \times \varnothing=\varnothing\). Similarly, \(\varnothing \times A=\varnothing\).
\end{proof}

\textbf{Exercise 1.4.6} Suppose \(A \neq \varnothing\) and \(B \neq \varnothing .\) Show \(A \times B=B \times A\) iff \(A = B\)
\begin{proof}
\(A \times B=B \times A \implies A = B\) :\\
Suppose \(A \times B=B \times A\) and \(A \neq B\). We need to prove this is contradictory. Since neither of $A$ or $B$ is empty, $A \neq B$ implies that $\exists a \in A: a \notin B$ or $\exists b \in B: b \notin A$. Since $A$ and $B$ are symmetric, we can suppose without loss of generality that $\exists a \in A: a \notin B$. Let $x$ be any arbitrary element in $B$, then \((a, x) \in A \times B\) (because $a \in A, x \in B$), while \((a, x) \notin B \times A\) (because $a \notin B$), so $A \times B \neq B \times A$. Contradiction. Hence, \(A \times B=B \times A \implies A = B\).
\\
\(A = B \implies A \times B=B \times A\) :\\
Given $A = B$, then $\forall (a, b) \in A \times B$ (in other words, $a \in A, b \in B$), we have $a \in B, b \in A$, so $(a, b) \in B \times A$. This implies that $A \subseteq B$. Similarly, $\forall (b, a) \in B \times A: (b, a) \in A \times B \implies A \supseteq B$. Hence, $A \subseteq B \text{ and } A \supseteq B \implies A = B$.
\end{proof}

\textbf{Exercise 1.6.5} Let \(n\) be an integer greater than or equal to \(2 .\) If \(a, b \in \mathbb{Z},\)
we say that \(a \sim b\) iff \(a-b\) is a multiple of \(n,\) that is, \(n\) divides \(a-b\).
\begin{proof}
1)Reflexive:\\
$a - a = 0$ is always divisable by $n$, so $a \sim a$ for all $a \in X$.\\
2)Symmetric:\\
For all $a, b \in X$, $a \sim b \iff n | (a - b) \iff n | (b - a) \iff b \sim a$\\
3)Transitive:\\
For all $a, b, c  \in X$, $a \sim b \text{ and } b \sim c \iff n | (a-b) \text{ and } n | (b-c) \iff n|((a-b)+(b-c)) \iff n|(a-c) \iff a \sim c$\\
Hence, the given relation is an equivalence relation.
\end{proof}

\textbf{Exercise 1.6.10} Show that, if \(c | a\) and \(c | b,\) and \(s, t \in \mathbb{Z},\) then \(c |(s a+t b)\).
\begin{proof}
$c|a \implies \exists n \in \bbZ: nc=a$\qquad$c|b \implies \exists m \in \bbZ: mc=b$\\
Therefore, $sa+tb=snc+tmc=(sn+tm)c$ where $(sn+tm) \in \bbZ$, so $c|(sa+tb)$.
\end{proof}

\textbf{Exercise 1.6.15} Show that \(\sim\) is an equivalence relation on $F=\{(a, b) \mid a, b \in \bbZ \text{ and } b \neq 0 \} $.
\begin{proof}
1)Reflexive:\\
For all $(a,b) \in F$:\space$(a,b)\sim(a,b)$ because $ab=ba$.\\
2)Symmetric:\\
For all $(a,b), (c,d) \in F$:\space $(a,b)\sim(c,d) \iff ad=bc \iff cb=da \iff (c,d)\sim(a,b)$\\
3)Transitive:\\
For all $(a,b), (c,d),(e,f) \in F$:\space $(a,b)\sim(c,d) \text{ and } (c,d)\sim(e,f) \iff ad=bc \text{ and } cf=de \iff \frac{d}{c}=\frac{b}{a}=\frac{f}{e} \iff af=be \iff (a,b)\sim(e,f)$\\
Hence, the given relation is an equivalence relation.
\end{proof}
% \begin{definition}
% We use
% \begin{verbatim}
% \begin{tag}
% Stuff
% \end{tag}
% \end{verbatim}
% environments for various things where tag = definition, lemma, equation, array, theorem.  There are many more.  Look in the header of the .tex file for more.
% \end{definition}

% If we have an equation, then it becomes pulled out of text.
% \begin{equation*}
% x + y = z.
% \end{equation*}

% Use dollar signs for math symbols not in equations $x+y = z$.

% Its easy to add special symbols such as $\alpha, \pm, \sum, x \in A$.

% If you want to find a special symbol look online.





\end{document}