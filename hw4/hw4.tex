\documentclass[12pt]{article}


%----------Packages----------
\usepackage{amsmath}
\usepackage{amssymb}
\usepackage{amsthm}
%\usepackage{amsrefs}
\usepackage{dsfont}
\usepackage{mathrsfs}
\usepackage{stmaryrd}
\usepackage[all]{xy}
\usepackage[mathcal]{eucal}
\usepackage{verbatim}  %%includes comment environment
\usepackage{fullpage}  %%smaller margins
\usepackage{hyperref}
\usepackage{setspace}
\onehalfspacing
%----------Commands----------

%%penalizes orphans
\clubpenalty=9999
\widowpenalty=9999

%% bold math capitals
\newcommand{\bA}{\mathbf{A}}
\newcommand{\bB}{\mathbf{B}}
\newcommand{\bC}{\mathbf{C}}
\newcommand{\bD}{\mathbf{D}}
\newcommand{\bE}{\mathbf{E}}
\newcommand{\bF}{\mathbf{F}}
\newcommand{\bG}{\mathbf{G}}
\newcommand{\bH}{\mathbf{H}}
\newcommand{\bI}{\mathbf{I}}
\newcommand{\bJ}{\mathbf{J}}
\newcommand{\bK}{\mathbf{K}}
\newcommand{\bL}{\mathbf{L}}
\newcommand{\bM}{\mathbf{M}}
\newcommand{\bN}{\mathbf{N}}
\newcommand{\bO}{\mathbf{O}}
\newcommand{\bP}{\mathbf{P}}
\newcommand{\bQ}{\mathbf{Q}}
\newcommand{\bR}{\mathbf{R}}
\newcommand{\bS}{\mathbf{S}}
\newcommand{\bT}{\mathbf{T}}
\newcommand{\bU}{\mathbf{U}}
\newcommand{\bV}{\mathbf{V}}
\newcommand{\bW}{\mathbf{W}}
\newcommand{\bX}{\mathbf{X}}
\newcommand{\bY}{\mathbf{Y}}
\newcommand{\bZ}{\mathbf{Z}}

%% blackboard bold math capitals
\newcommand{\bbA}{\mathbb{A}}
\newcommand{\bbB}{\mathbb{B}}
\newcommand{\bbC}{\mathbb{C}}
\newcommand{\bbD}{\mathbb{D}}
\newcommand{\bbE}{\mathbb{E}}
\newcommand{\bbF}{\mathbb{F}}
\newcommand{\bbG}{\mathbb{G}}
\newcommand{\bbH}{\mathbb{H}}
\newcommand{\bbI}{\mathbb{I}}
\newcommand{\bbJ}{\mathbb{J}}
\newcommand{\bbK}{\mathbb{K}}
\newcommand{\bbL}{\mathbb{L}}
\newcommand{\bbM}{\mathbb{M}}
\newcommand{\bbN}{\mathbb{N}}
\newcommand{\bbO}{\mathbb{O}}
\newcommand{\bbP}{\mathbb{P}}
\newcommand{\bbQ}{\mathbb{Q}}
\newcommand{\bbR}{\mathbb{R}}
\newcommand{\bbS}{\mathbb{S}}
\newcommand{\bbT}{\mathbb{T}}
\newcommand{\bbU}{\mathbb{U}}
\newcommand{\bbV}{\mathbb{V}}
\newcommand{\bbW}{\mathbb{W}}
\newcommand{\bbX}{\mathbb{X}}
\newcommand{\bbY}{\mathbb{Y}}
\newcommand{\bbZ}{\mathbb{Z}}

%% script math capitals
\newcommand{\sA}{\mathscr{A}}
\newcommand{\sB}{\mathscr{B}}
\newcommand{\sC}{\mathscr{C}}
\newcommand{\sD}{\mathscr{D}}
\newcommand{\sE}{\mathscr{E}}
\newcommand{\sF}{\mathscr{F}}
\newcommand{\sG}{\mathscr{G}}
\newcommand{\sH}{\mathscr{H}}
\newcommand{\sI}{\mathscr{I}}
\newcommand{\sJ}{\mathscr{J}}
\newcommand{\sK}{\mathscr{K}}
\newcommand{\sL}{\mathscr{L}}
\newcommand{\sM}{\mathscr{M}}
\newcommand{\sN}{\mathscr{N}}
\newcommand{\sO}{\mathscr{O}}
\newcommand{\sP}{\mathscr{P}}
\newcommand{\sQ}{\mathscr{Q}}
\newcommand{\sR}{\mathscr{R}}
\newcommand{\sS}{\mathscr{S}}
\newcommand{\sT}{\mathscr{T}}
\newcommand{\sU}{\mathscr{U}}
\newcommand{\sV}{\mathscr{V}}
\newcommand{\sW}{\mathscr{W}}
\newcommand{\sX}{\mathscr{X}}
\newcommand{\sY}{\mathscr{Y}}
\newcommand{\sZ}{\mathscr{Z}}

\renewcommand{\phi}{\varphi}
%\renewcommand{\emptyset}{\O}

\providecommand{\abs}[1]{\lvert #1 \rvert}
\providecommand{\norm}[1]{\lVert #1 \rVert}
\providecommand{\x}{\times}
\providecommand{\ar}{\rightarrow}
\providecommand{\arr}{\longrightarrow}


%----------Theorems----------

\newtheorem{theorem}{Theorem}[section]
\newtheorem{proposition}[theorem]{Proposition}
\newtheorem{lemma}[theorem]{Lemma}
\newtheorem{corollary}[theorem]{Corollary}

\theoremstyle{definition}
\newtheorem{definition}[theorem]{Definition}
\newtheorem{nondefinition}[theorem]{Non-Definition}
\newtheorem{exercise}[theorem]{Exercise}

\numberwithin{equation}{subsection}


%----------Title-------------
\title{Homework 4}
\author{Yifan}

\begin{document}

\pagestyle{plain}


%%---  sheet number for theorem counter
%\setcounter{section}{1}

\begin{center}
{\large Homework 4} \\
\vspace{.2in}
Yifan Liu
\end{center}

\bigskip \bigskip

% \section{Chapter 1}

\textbf{Exercise 1.7.18} Show that \(f^{-1} \circ f=I_{A}\) and \(f \circ f^{-1}=I_{B}\)
\begin{proof}
% \(f^{-1} \circ f=I_{A}\):\\
Given $f^{-1}$ exists, then $\forall a \in A, \exists b \in B$ s.t. $f(a)=b$ and $f^{-1}(b)=a$. Therefore, $\forall a \in A, (f^{-1} \circ f)(a)=f^{-1} (f(a)) = f^{-1}(b) = a$, which implies that \(f^{-1} \circ f=I_{A}\). In the same way, $\forall b \in B, (f \circ f^{-1})(b)=f (f^{-1}(b)) = f^{-1}(a) = b$, in other words, \(f \circ f^{-1}=I_{B}\).
\end{proof}

\textbf{Exercise 1.7.19} Suppose \(A, B,\) and \(C\) are sets and \(f: A \rightarrow B\) and \(g: B \rightarrow\)\(C\) are bijections. Show that \(g \circ f\) is a bijection. Compute \((g \circ f)^{-1}: C \rightarrow A\).
\begin{proof}
$\forall a \in A, \exists b \in B$ s.t. $f(a)=b$, and $\forall b \in B, \exists c \in C$ s.t. $g(b)=c$. So $g \circ f: A\rightarrow C$ is such that $(g\circ f)(a)=c$.


Injective:
If $(g\circ f)(a)=(g\circ f)(a')=c$, then $g(f(a))=g(f(a'))=c$. Let $b=g^{-1}(c)$, which is unique in $B$. Then $g(f(a))=g(f(a'))=g(b)=c$, so $f(a)=f(a')=b$. Because $f$ is bijective, we have $a=a'$.

Surjective:
$\forall c \in C$, because there is a bijective $g$, $\exists b \in B$ such that $g(b)=c$. Meanwhile, because $f$ is bijective, $\exists a \in A$ such that $f(a)=b$. So for any $c$, there is such an $a$ that $(g\circ f)(a)=c$.

Hence, $g\circ f$ is bijective.

Therefore, if $(g\circ f)(a)=c$ and $(g\circ f)^{-1}(c)=a$, then $\exists b \in B$ such that  $(a, b) \in f \text { and }(b, c) \in g \text { and } (b, a) \in f^{-1} \text { and }(c, b) \in g^{-1}$. Hence,
$$(g\circ f)^{-1}(c)=a=f^{-1}(b)=f^{-1}(g^{-1}(c))=(f^{-1}\circ g^{-1})(c)$$
In other words, $(g \circ f)^{-1}=f^{-1} \circ g^{-1}$
% $$
% \begin{aligned}(g \circ f)^{-1} &=\{(c, a): \exists b \in B \text{ s.t. }(a, b) \in f \text { and }(b, c) \in g\} \\ &=\left\{(c, a): \exists b \in B \text{ s.t. }(b, a) \in f^{-1} \text { and }(c, b) \in g^{-1}\right\} \\ &=f^{-1} \circ g^{-1} \end{aligned}
% $$
\end{proof}

\textbf{Exercise 1.7.22} Define \(f: \mathbb{N} \rightarrow \mathbb{Z}\) by
$$
f(n)=\left\{\begin{array}{ll}{\frac{n}{2},} & {\text { if } n \text { is even }} \\ {\frac{1-n}{2},} & {\text { if } n \text { is odd. }}\end{array}\right.
$$
Show that \(f\) is a bijection.
\begin{proof}
Injective:\\
If $f(a)=f(a')=m$, then:

If $m>0$, then $a$ and $a'$ must be even. This is because if $a \in \bbN$ (which implies that $a>0$) is odd, then $f(a) = \frac{1-a}{2} <0 $. The same is for $a'$. Therefore, $f(a)=f(a') \implies \frac{a}{2} = \frac{a'}{2} \implies a=a'$.

If $m\leq0$, then $a$ and $a'$ must be odd. This is because if $a \in \bbN$ (which implies that $a>0$) is even, then $f(a) = \frac{a}{2} >0 $. The same is for $a'$. Therefore, $f(a)=f(a') \implies \frac{1-a}{2} = \frac{1-a'}{2} \implies 1-a=1-a'\implies a=a'$.
\\
Therefore, $f(a)=f(a') \implies a=a'$. $f$ is injective.
\\
\smallskip\\
Surjective:\\
For any $m \in \bbZ$:

If $m>0$, we have $n=2m \in \bbN$ such that $f(n)=\frac{n}{2}=m$.

If $m\leq0$, we have $n=-2m+1 \in \bbN$ such that $f(n)=\frac{1-(-2m+1)}{2}=m$.
\\
Therefore, $f$ is surjective.\\
\smallskip\\
Since $f$ is both injective and surjective, $f$ is therefore bijective.
\end{proof}

\textbf{Exercise 1.8.4} Show that the following are finite sets: (i) The English alphabet.
(ii) The set of all possible twelve letter words made up of letters from the English alphabet.
(iii) The set of all subsets of a finite set.
\begin{proof}
(i) Let a bijection $f$ map ``a'' to 1, ``b'' to 2, ... , ``z'' to 26. Then there is a bijection between the English alphabet and $\{1,2,...,26\}$, so the English alphabet is finite.\newline
(ii) Let $f$ be the bijection in (i). If the 12-letter word is $a_{1}a_{2}a_{3}...a_{12}$ where each $a_{i}$ is an English letter, then it can be mapped to $1+\sum_{i=1}^{12} 26^{i-1}\times (f(a_{i})-1)$. So there is a bijection between the set of all possible twelve letter words and the set $\{1,2,...,26^{12}\}$, which implies it's finite.
\\(If the ``words'' in the question means words that exist in real life, we can prove this by showing that this set is a proper subset of the set mentioned above (all random combinations of the letters) and is therefore also finite.)\\
(iii) Suppose the finite set $X$ has $n$ elements, then there exists a bijection of this set to $\{1,2,...,n\}$. Let's call the element that corresponds to $i$ ``the $i$th element in $X$''. Construct a binary number ${(b_{n}b_{n-2}...b_2b_1)}_{2}$ for each subset of this finite set $A$ in such a way: if the $i$th element in $X$ is also in $A$, then let $b_i$ be 1, and otherwise 0. Since the binary number can be written in a decimal form, the set of all subsets has a bijection to $\{0,1,2,...,2^n-1\}$, and therefore also a bijection to $\{1,2,...,2^n\}$. Hence, the set of all subsets of a finite set is finite.
\end{proof}

\textbf{Exercise 1.8.12} Suppose \(A, B,\) and \(C\) are subsets of a set \(X\) such that
\(A \subseteq B \subseteq C .\) Show that if \(A\) and \(C\) have the same cardinality, then \(A\) and
\(B\) have the same cardinality.
\begin{proof}
Because $A$ and $C$ have the same cardinality, there exists a bijection $h: A \rightarrow C$.

Now we prove $A$ and $B$ have the same cardinality, in other words, there is a bijection between $A$ and $B$.

Injection $f:A\rightarrow B$. $f(x)=x, x\in A$. This is because $A \subseteq B$. This is apparently an injection.

Injection $g:B\rightarrow A$. $g(x)=h^{-1}(x)$. Because $B \subseteq C$, and there exists bijection $h^{-1}(x)$ for any $x \in C$, $g(x)$ is definitely injective.

Therefore, according to Schr\"oder-Bernstein Theorem, there exists a bijection between $A$ and $B$, so \(A\) and
\(B\) have the same cardinality.
\end{proof}

\textbf{Exercise 6(a)} Construct an injection from \(\mathbb{N}\) to \(\mathbb{N} \times \mathbb{N}\)
$$
f(x) = (x,x)
$$
\begin{proof}
If $(a,a)=(b,b)$, apparently $a=b$. This is an injection.

\end{proof}

\textbf{Exercise 6(b)} Construct an injection from \(\mathbb{N} \times \mathbb{N}\) to \(\mathbb{N}\)
$$
f(x,y)=2^x3^y
$$
\begin{proof}
If $f(a,b) = f(c, d)$, then $2^a3^b = 2^c3^d$. According to the unique factorization theorom, $a = c$ and $b =d$, so $(a,b)=(c,d)$. This is an injection.
\end{proof}

\textbf{Exercise 6(c)} Construct an injection from \(\mathbb{N}\) to \(\mathbb{Q} \)
$$
f(x)=x
$$
\begin{proof}
This is apparent because $\bbN \subseteq\bbQ$.
\end{proof}

\textbf{Exercise 6(d)} Construct an injection from \(\mathbb{Q}\) to \(\mathbb{N} \times \mathbb{N}\)
$$
f(\frac{a}{b})=\left\{\begin{array}{ll}{(2a+1,b),} & {\text { if } a \geq0} \\ {(-2a, b),} & {\text { if } a <0}\end{array}\right.( \frac{a}{b} \text{ in lowest form, }b>0)
$$
\begin{proof}
Apparently $2a+1$ (in case of $a\geq 0$) or $-2a$ (in case of $a<0$) and b are in $\bbN$.\\
If $f(\frac{a}{b})=(x,y)$, $b=y$, $x$ odd $\implies 2a+1=x$ ; $x$ even $\implies -2a=x$. Hence, $a,b$ unique, $f$ is injection.
\end{proof}

\textbf{Exercise 6(e)} Use the above and 1.8.12 to conclude that $|\bbQ| = |\bbN|$
\begin{proof}
Let $f$ in this question be the $f$ constructed in 6(d).\\
Let $S=\{f(n)|n\in \bbN\}$, apparently there is a bijection between $\bbN$ and $S$, and $|\bbN|=|S|$.
Let $T=\{f(x)|x\in\bbQ\}$, apparently there is a bijection between $\bbQ$ and $T$, and $|\bbQ|=|T|$. Also, $T \subseteq \bbN \times\bbN$.
Besides, because $\bbN \subseteq \bbQ$, $S \subseteq T$.
\\
There is an injection from $\bbN$ to $\bbN \times \bbN$, and an injection from $\bbN \times \bbN$ to $\bbN$, so $|\bbN|=|\bbN \times \bbN|$. Hence $|S|=|\bbN \times \bbN|$.
\\
Therefore, since $S \subseteq T \subseteq \bbN \times \bbN$, according to 1.8.12, $S$ and $T$ have the same cardinality. This implies that $\bbN$ and $\bbQ$ have the same cardinality, or $|\bbQ| = |\bbN|$.
\end{proof}


\end{document}